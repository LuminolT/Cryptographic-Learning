\chapter{第一次作业}

\newpage
\chapter{第二次作业}

\section{完美安全性}

\textbf{题目}:请证明或者提出反例:对于任意的加密方案,若密钥空间大小和明文空
间大小相等,且每次密钥均从密钥空间中按照均匀分布随机选取,那么方案
在已知密文攻击下满足完美安全性(perfect secrecy)。

\textbf{解答}:给定特殊的加密方案,明文空间、密文空间和密钥空间满足:
\begin{equation}
    \mathcal{M}=\mathcal{C}=\mathcal{K}=\left\{0, 1\right\}
\end{equation}

加密函数为:
\begin{equation}
    \mathsf{Enc}_K(m) = m
\end{equation}

解密函数为:
\begin{equation}
    \mathsf{Dec}_K(m) = m
\end{equation}

则此时显然有:
\begin{equation}
    \mathsf{Pr}\left[\mathsf{Enc}_K(m) = 1\right] = \left\{
        \begin{aligned}
            &0, & m = 0\\
            &1, & m = 1
        \end{aligned}
    \right.
\end{equation}

因此对$\forall m, m' \in \mathcal{M}, \exists c = 1 \in \mathcal{C}$使得:
\begin{equation}
    \mathsf{Pr}\left[\mathsf{Enc}_K(m) = c\right] \neq \mathsf{Pr}\left[\mathsf{Enc}_K(m') = c\right]
\end{equation}

因此该加密方案不满足完美安全性,原命题不成立。

\section{3级线性反馈移位寄存器}

\textbf{题目}:已知流密码的密文串 1010110110 和对应的明文串
0100010001,且知道密钥流是使用 3 级线性反馈移位寄存器产生的,试破译
该密码系统。

\textbf{解答}:由于密钥流是由3级LFSR产生,且明文串$msg=0100010001$,密文串$cipher=1010110110$
因此密钥流(即为一段状态流)为:
\begin{equation}
    a = msg \oplus cipher = 1110100111
\end{equation}

则可根据状态流前$6$个bit建立如下方程:
\begin{equation}
    \left[
        \begin{matrix}
            a_4&a_5&a_6
        \end{matrix}
    \right]
    =
    \left[
        \begin{matrix}
            c_3&c_2&c_1
        \end{matrix}
    \right]
    \left[
        \begin{matrix}
            a_1 & a_2 & a_3\\
            a_2 & a_3 & a_4\\
            a_3 & a_4 & a_5\\
        \end{matrix}
    \right]
\end{equation}

代入$a$得到:
\begin{equation}
    \left[
        \begin{matrix}
            0&1&0
        \end{matrix}
    \right]
    =
    \left[
        \begin{matrix}
            c_3&c_2&c_1
        \end{matrix}
    \right]
    \left[
        \begin{matrix}
            1 &1 &1\\
            1 &1 &0\\
            1 &0 &1\\
        \end{matrix}
    \right]
\end{equation}

因此可得到下式(计算通过Matlab):
\begin{equation}
    \left[
        \begin{matrix}
            c_3&c_2&c_1
        \end{matrix}
    \right]
    =
    \left[
        \begin{matrix}
            0&1&0
        \end{matrix}
    \right]
    \left[
        \begin{matrix}
            1 &1 &1\\
            1 &1 &0\\
            1 &0 &1\\
        \end{matrix}
    \right]^{-1}
    =    \left[
        \begin{matrix}
            1&0&1
        \end{matrix}
    \right]
\end{equation}

又由序列递推关系:
\begin{equation}
    a_{h+3} = c_1 a_{h+2} \oplus c_2 a_{h+1} \oplus c_3 a_{h}
\end{equation}

得到密钥流的递推关系为:
\begin{equation}
    a_{h+3} = a_{h+2} \oplus a_{h}
\end{equation}

\section{DES穷尽搜索}

\textbf{题目}:1) 设$M^{\prime}$是$M$的逐比特取补,证明在 DES 中,如果对明文分组和加
密密钥都逐比特取补,那么得到的密文也是原密文的逐比特取补,即:

如果 $Y = DES(k, X)$

那么 $Y^{\prime} = DES(k^{\prime}, X^{\prime})$。

\textbf{解答}:由于在DES中置换、密钥分组无论输入如何都有相同的表现,因此在输入中被取补的bit将在
输出中被取补。因此只需要考虑DES中的16轮迭代运算。

假设$i$轮后得到$L_i^{\prime}, R_i^{\prime}$,则其上一轮为$L_{i-1}^{\prime}, R_{i-1}^{\prime}$
并使用密钥$k_{i-1}^{\prime}$加密。

由$L_i=R_{i-1}$,可得:
\begin{equation}
    L_i^{\prime} = R_{i-1}^{\prime}
\end{equation}

由$R_i = L_{i-1} \oplus f(R_{i-1}, k_{i-1})$可得:
\begin{equation}
    R_i^{\prime} = L_{i-1}^{\prime} \oplus f(R_{i-1}^{\prime}, k_{i-1}^{\prime})
\end{equation}

由于在轮函数$f$中两参数异或,有$f(R,k)=f(R^{\prime},k^{\prime})$,因此进一步化为:
\begin{equation}
    \begin{aligned}
        R_i^{\prime}    &= L_{i-1}^{\prime} \oplus f(R_{i-1}, k_{i-1})\\
                        &= (L_{i-1} \oplus f(R_{i-1}, k_{i-1}))^{\prime}\\
                        &= R_i
    \end{aligned}
\end{equation}

因此在16轮迭代后得到的bit串恰为原来的补,而开始和最后的置换不改变补关系,因此有:
\begin{equation}
    Y^{\prime} = DES(k^{\prime}, X^{\prime})
\end{equation}

证毕。

\textbf{题目}:2) 对 DES 进行穷尽搜索攻击时,需要在由$2^{56}$个密钥构成的密钥空间
进行。能否根据上述结论减小搜索所用的密钥空间大小,进而优化穷尽搜索
效率。

\textbf{解答}:由上述结论可知,在进行CPA时只需将原来的判断条件$y = DES(k, x)$更改为
$y = DES(k, x)~or~y = DES(k, x)^{\prime}$即可。

根据补运算的性质,可将原密钥空间$\mathcal{K}$进行划分,划分后的子空间$\mathcal{K}_1,\mathcal{K}_2$满足:
\begin{itemize}
    \item $\mathcal{K}_1\cup \mathcal{K}_2 = \mathcal{K}$
    \item $|\mathcal{K}_1|=|\mathcal{K}_2|$
    \item $\forall k_1, k_2 \in \mathcal{K}_i, i = 0, 1$满足$k_1 \neq k_2^{\prime}$
    \item $\forall k_1 \in \mathcal{K}_i, i = 0, 1$,$\exists k_2 \in \mathcal{K}_{i^\prime}$使得$k_1 = k_2^{\prime}$
\end{itemize}

因此优化后的密钥空间大小$|\mathcal{K}_1| = |\mathcal{K}| / 2 = 2^{55}$

\newpage

\chapter{第三次作业}

\section{伪随机函数与伪随机数生成器}

\textbf{题目}:令$G$是一个安全的伪随机数生成器(Pseudo Random Generator),
$G_0(s)$表示当输入$s \in\{0, 1\}^n$时$G$的输出。定义函数$F(k, x) = G_0(k) \oplus x$。
证明$F$不是一个伪随机函数(Pseudo Random Function)。

\begin{Solution}
    根据异或运算性质得到给定PRF有$F:\mathcal{K}\times X \rightarrow X$
    
    下面根据PRF定义模型,构造攻击使得其优势不可忽略。

    取$\forall x_1, x_2 \in X, x_1 \neq x_2$,

    在一次实验中给定两个质询$x_1, x_2$,得到$y_1=f(x_1), y_2=f(x_2)$。

    若$b=0$则$f:=F(k,x) = G_0(k) \oplus x$,则有:
    \begin{equation}
        \begin{aligned}
            y_1 \oplus y_2  &= G_0(k) \oplus x_1 \oplus G_0(k) \oplus x_2\\
                            &= x_1 \oplus x_2
        \end{aligned}
    \end{equation}

    因此我们给定策略:
    
    当质询结果$y_1 \oplus y_2 = x_1 \oplus x_2$,认为是实验0(即$b=0$),
    反之则认为是实验1(即$b=1$,此时由真PRF产生结果)

    若$b=0$,可得到$\mathrm{Pr}[EXP(0)=1] = 1$

    若$b=1$,可得到$\mathrm{Pr}[EXP(1)=1] = 1/2^{|X|} = 1/2^{n}$
    
    计算优势:
    \begin{equation}
        \begin{aligned}
            \mathrm{Adv_{PRF}}[A, F]    &= |\mathrm{Pr}[EXP(0)=1] - \mathrm{Pr}[EXP(1)=1]|\\
                                        &= 1 - \frac{1}{2^n}
        \end{aligned}
    \end{equation}

    显然,该优势是不可忽略的,因此该PRF不安全,即$F$不是一个伪随机函数。
    
\end{Solution}


\newpage
\section{选择明文攻击}

\textbf{题目}:考虑 CBC 分组加密的一个变种:定义初始向量 IV 为逐渐递增的计数
器(而不是随机选取 IV)。证明上述变种不满足多次使用密钥场景下选择明
文攻击(CPA)安全性。\footnote{建议参考 ppt 第 41 页}

\begin{Solution}
    参考PPT中的攻击方式:

    第一次质询:发送$m_0=m_1=0\in X$,得到相同质询结果:
    \begin{equation}
        c_1 = [IV_1, E(k, 0\oplus IV_1)] = [IV_1, E(k, IV_1)]
    \end{equation}
    由于 IV采用计数器选取,可预测下一次质询的初始向量为:
    \begin{equation}
        IV_2 = IV_1 + 1
    \end{equation}

    第二次质询:发送$m_0 = IV_1 \oplus IV_2, m_1 \neq m_0$

    得到质询结果为:
    \begin{equation}
        c_2 = \left\{
        \begin{aligned}
            &[IV_2, E(k,m_0 \oplus IV_2)] = [IV_2, E(k, IV_1)] &, case0\\
            &[IV_2, E(k,m_1\oplus IV_2)]    &,case1
        \end{aligned}
        \right.
    \end{equation}

    观察发现$case0$中$c_1[1] = c_2[1]$,因此可判断,当第二次质询结果$c_2[1]=c_1[1]$时,
    认为是实验0(即$b=0$),反之则认为是实验1(即$b=1$)。

    若$b=0$,可得到$\mathrm{Pr}[EXP(0)=1] = 1$

    若$b=1$,由于选取$m_1 \neq m_0$,加密结果必然不同,可得到$\mathrm{Pr}[EXP(1)=1] = 0$

    计算优势:
    \begin{equation}
        \begin{aligned}
            \mathrm{Adv_{CBC'}}[A, E_{\mathrm{CBC}}]    &= |\mathrm{Pr}[EXP(0)=1] - \mathrm{Pr}[EXP(1)=1]|\\
                                        &= 1
        \end{aligned}
    \end{equation}

    显然,该优势是不可忽略的,因此该加密方案不满足多次使用密钥场景下的CPA安全性。
    
\end{Solution}