\chapter{Week 2}

\section{完美安全性}

\textbf{题目}:请证明或者提出反例:对于任意的加密方案,若密钥空间大小和明文空
间大小相等,且每次密钥均从密钥空间中按照均匀分布随机选取,那么方案
在已知密文攻击下满足完美安全性(perfect secrecy)。

\textbf{解答}:给定特殊的加密方案,明文空间、密文空间和密钥空间满足:
\begin{equation}
    \mathcal{M}=\mathcal{C}=\mathcal{K}=\left\{0, 1\right\}
\end{equation}

加密函数为:
\begin{equation}
    \mathsf{Enc}_K(m) = m
\end{equation}

解密函数为:
\begin{equation}
    \mathsf{Dec}_K(m) = m
\end{equation}

则此时显然有:
\begin{equation}
    \mathsf{Pr}\left[\mathsf{Enc}_K(m) = 1\right] = \left\{
        \begin{aligned}
            &0, & m = 0\\
            &1, & m = 1
        \end{aligned}
    \right.
\end{equation}

因此对$\forall m, m' \in \mathcal{M}, \exists c = 1 \in \mathcal{C}$使得:
\begin{equation}
    \mathsf{Pr}\left[\mathsf{Enc}_K(m) = c\right] \neq \mathsf{Pr}\left[\mathsf{Enc}_K(m') = c\right]
\end{equation}

因此该加密方案不满足完美安全性,原命题不成立。