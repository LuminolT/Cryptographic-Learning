\chapter{第一次作业}

\newpage
\chapter{第二次作业}

\section{完美安全性}

\textbf{题目}:请证明或者提出反例:对于任意的加密方案,若密钥空间大小和明文空
间大小相等,且每次密钥均从密钥空间中按照均匀分布随机选取,那么方案
在已知密文攻击下满足完美安全性(perfect secrecy)。

\textbf{解答}:给定特殊的加密方案,明文空间、密文空间和密钥空间满足:
\begin{equation}
    \mathcal{M}=\mathcal{C}=\mathcal{K}=\left\{0, 1\right\}
\end{equation}

加密函数为:
\begin{equation}
    \mathsf{Enc}_K(m) = m
\end{equation}

解密函数为:
\begin{equation}
    \mathsf{Dec}_K(m) = m
\end{equation}

则此时显然有:
\begin{equation}
    \mathsf{Pr}\left[\mathsf{Enc}_K(m) = 1\right] = \left\{
        \begin{aligned}
            &0, & m = 0\\
            &1, & m = 1
        \end{aligned}
    \right.
\end{equation}

因此对$\forall m, m' \in \mathcal{M}, \exists c = 1 \in \mathcal{C}$使得:
\begin{equation}
    \mathsf{Pr}\left[\mathsf{Enc}_K(m) = c\right] \neq \mathsf{Pr}\left[\mathsf{Enc}_K(m') = c\right]
\end{equation}

因此该加密方案不满足完美安全性,原命题不成立。

\section{3级线性反馈移位寄存器}

\textbf{题目}:已知流密码的密文串 1010110110 和对应的明文串
0100010001,且知道密钥流是使用 3 级线性反馈移位寄存器产生的,试破译
该密码系统。

\textbf{解答}:由于密钥流是由3级LFSR产生,且明文串$msg=0100010001$,密文串$cipher=1010110110$
因此密钥流(即为一段状态流)为:
\begin{equation}
    a = msg \oplus cipher = 1110100111
\end{equation}

则可根据状态流前$6$个bit建立如下方程:
\begin{equation}
    \left[
        \begin{matrix}
            a_4&a_5&a_6
        \end{matrix}
    \right]
    =
    \left[
        \begin{matrix}
            c_3&c_2&c_1
        \end{matrix}
    \right]
    \left[
        \begin{matrix}
            a_1 & a_2 & a_3\\
            a_2 & a_3 & a_4\\
            a_3 & a_4 & a_5\\
        \end{matrix}
    \right]
\end{equation}

代入$a$得到:
\begin{equation}
    \left[
        \begin{matrix}
            0&1&0
        \end{matrix}
    \right]
    =
    \left[
        \begin{matrix}
            c_3&c_2&c_1
        \end{matrix}
    \right]
    \left[
        \begin{matrix}
            1 &1 &1\\
            1 &1 &0\\
            1 &0 &1\\
        \end{matrix}
    \right]
\end{equation}

因此可得到下式(计算通过Matlab):
\begin{equation}
    \left[
        \begin{matrix}
            c_3&c_2&c_1
        \end{matrix}
    \right]
    =
    \left[
        \begin{matrix}
            0&1&0
        \end{matrix}
    \right]
    \left[
        \begin{matrix}
            1 &1 &1\\
            1 &1 &0\\
            1 &0 &1\\
        \end{matrix}
    \right]^{-1}
    =    \left[
        \begin{matrix}
            1&0&1
        \end{matrix}
    \right]
\end{equation}

又由序列递推关系:
\begin{equation}
    a_{h+3} = c_1 a_{h+2} \oplus c_2 a_{h+1} \oplus c_3 a_{h}
\end{equation}

得到密钥流的递推关系为:
\begin{equation}
    a_{h+3} = a_{h+2} \oplus a_{h}
\end{equation}

\section{DES穷尽搜索}

\textbf{题目}:1) 设$M^{\prime}$是$M$的逐比特取补,证明在 DES 中,如果对明文分组和加
密密钥都逐比特取补,那么得到的密文也是原密文的逐比特取补,即:

如果 $Y = DES(k, X)$

那么 $Y^{\prime} = DES(k^{\prime}, X^{\prime})$。

\textbf{解答}:由于在DES中置换、密钥分组无论输入如何都有相同的表现,因此在输入中被取补的bit将在
输出中被取补。因此只需要考虑DES中的16轮迭代运算。

假设$i$轮后得到$L_i^{\prime}, R_i^{\prime}$,则其上一轮为$L_{i-1}^{\prime}, R_{i-1}^{\prime}$
并使用密钥$k_{i-1}^{\prime}$加密。

由$L_i=R_{i-1}$,可得:
\begin{equation}
    L_i^{\prime} = R_{i-1}^{\prime}
\end{equation}

由$R_i = L_{i-1} \oplus f(R_{i-1}, k_{i-1})$可得:
\begin{equation}
    R_i^{\prime} = L_{i-1}^{\prime} \oplus f(R_{i-1}^{\prime}, k_{i-1}^{\prime})
\end{equation}

由于在轮函数$f$中两参数异或,有$f(R,k)=f(R^{\prime},k^{\prime})$,因此进一步化为:
\begin{equation}
    \begin{aligned}
        R_i^{\prime}    &= L_{i-1}^{\prime} \oplus f(R_{i-1}, k_{i-1})\\
                        &= (L_{i-1} \oplus f(R_{i-1}, k_{i-1}))^{\prime}\\
                        &= R_i
    \end{aligned}
\end{equation}

因此在16轮迭代后得到的bit串恰为原来的补,而开始和最后的置换不改变补关系,因此有:
\begin{equation}
    Y^{\prime} = DES(k^{\prime}, X^{\prime})
\end{equation}

证毕。

\textbf{题目}:2) 对 DES 进行穷尽搜索攻击时,需要在由$2^{56}$个密钥构成的密钥空间
进行。能否根据上述结论减小搜索所用的密钥空间大小,进而优化穷尽搜索
效率。

\textbf{解答}:由上述结论可知,在进行CPA时只需将原来的判断条件$y = DES(k, x)$更改为
$y = DES(k, x)~or~y = DES(k, x)^{\prime}$即可。

根据补运算的性质,可将原密钥空间$\mathcal{K}$进行划分,划分后的子空间$\mathcal{K}_1,\mathcal{K}_2$满足:
\begin{itemize}
    \item $\mathcal{K}_1\cup \mathcal{K}_2 = \mathcal{K}$
    \item $|\mathcal{K}_1|=|\mathcal{K}_2|$
    \item $\forall k_1, k_2 \in \mathcal{K}_i, i = 0, 1$满足$k_1 \neq k_2^{\prime}$
    \item $\forall k_1 \in \mathcal{K}_i, i = 0, 1$,$\exists k_2 \in \mathcal{K}_{i^\prime}$使得$k_1 = k_2^{\prime}$
\end{itemize}

因此优化后的密钥空间大小$|\mathcal{K}_1| = |\mathcal{K}| / 2 = 2^{55}$

\newpage

\chapter{第三次作业}

\section{伪随机函数与伪随机数生成器}

\textbf{题目}:令$G$是一个安全的伪随机数生成器(Pseudo Random Generator),
$G_0(s)$表示当输入$s \in\{0, 1\}^n$时$G$的输出。定义函数$F(k, x) = G_0(k) \oplus x$。
证明$F$不是一个伪随机函数(Pseudo Random Function)。

\begin{Solution}
    根据异或运算性质得到给定PRF有$F:\mathcal{K}\times X \rightarrow X$
    
    下面根据PRF定义模型,构造攻击使得其优势不可忽略。

    取$\forall x_1, x_2 \in X, x_1 \neq x_2$,

    在一次实验中给定两个质询$x_1, x_2$,得到$y_1=f(x_1), y_2=f(x_2)$。

    若$b=0$则$f:=F(k,x) = G_0(k) \oplus x$,则有:
    \begin{equation}
        \begin{aligned}
            y_1 \oplus y_2  &= G_0(k) \oplus x_1 \oplus G_0(k) \oplus x_2\\
                            &= x_1 \oplus x_2
        \end{aligned}
    \end{equation}

    因此我们给定策略:
    
    当质询结果$y_1 \oplus y_2 = x_1 \oplus x_2$,认为是实验0(即$b=0$),
    反之则认为是实验1(即$b=1$,此时由真PRF产生结果)

    若$b=0$,可得到$\mathrm{Pr}[EXP(0)=1] = 1$

    若$b=1$,可得到$\mathrm{Pr}[EXP(1)=1] = 1/2^{|X|} = 1/2^{n}$
    
    计算优势:
    \begin{equation}
        \begin{aligned}
            \mathrm{Adv_{PRF}}[A, F]    &= |\mathrm{Pr}[EXP(0)=1] - \mathrm{Pr}[EXP(1)=1]|\\
                                        &= 1 - \frac{1}{2^n}
        \end{aligned}
    \end{equation}

    显然,该优势是不可忽略的,因此该PRF不安全,即$F$不是一个伪随机函数。
    
\end{Solution}


\newpage
\section{选择明文攻击}

\textbf{题目}:考虑 CBC 分组加密的一个变种:定义初始向量 IV 为逐渐递增的计数
器(而不是随机选取 IV)。证明上述变种不满足多次使用密钥场景下选择明
文攻击(CPA)安全性。\footnote{建议参考 ppt 第 41 页}

\begin{Solution}
    参考PPT中的攻击方式:

    第一次质询:发送$m_0=m_1=0\in X$,得到相同质询结果:
    \begin{equation}
        c_1 = [IV_1, E(k, 0\oplus IV_1)] = [IV_1, E(k, IV_1)]
    \end{equation}
    由于 IV采用计数器选取,可预测下一次质询的初始向量为:
    \begin{equation}
        IV_2 = IV_1 + 1
    \end{equation}

    第二次质询:发送$m_0 = IV_1 \oplus IV_2, m_1 \neq m_0$

    得到质询结果为:
    \begin{equation}
        c_2 = \left\{
        \begin{aligned}
            &[IV_2, E(k,m_0 \oplus IV_2)] = [IV_2, E(k, IV_1)] &, case0\\
            &[IV_2, E(k,m_1\oplus IV_2)]    &,case1
        \end{aligned}
        \right.
    \end{equation}

    观察发现$case0$中$c_1[1] = c_2[1]$,因此可判断,当第二次质询结果$c_2[1]=c_1[1]$时,
    认为是实验0(即$b=0$),反之则认为是实验1(即$b=1$)。

    若$b=0$,可得到$\mathrm{Pr}[EXP(0)=1] = 1$

    若$b=1$,由于选取$m_1 \neq m_0$,加密结果必然不同,可得到$\mathrm{Pr}[EXP(1)=1] = 0$

    计算优势:
    \begin{equation}
        \begin{aligned}
            \mathrm{Adv_{CBC'}}[A, E_{\mathrm{CBC}}]    &= |\mathrm{Pr}[EXP(0)=1] - \mathrm{Pr}[EXP(1)=1]|\\
                                        &= 1
        \end{aligned}
    \end{equation}

    显然,该优势是不可忽略的,因此该加密方案不满足多次使用密钥场景下的CPA安全性。
    
\end{Solution}

\newpage
\chapter{第四次作业}

\section{密钥交换}
Alice 和 Bob 拥有同一个公开社交平台的账号(如,微博)并且知道彼
此的 id, 他们想要利用公开平台进行秘密通信。

\begin{enumerate}
    \item 请设计一个安全的通信方法。
    \item 请分析你设计的方案能否抵御中间人攻击。
\end{enumerate}

\begin{Solution}
    (1) 假设该公开社交平台服务器作为证书管理机构CA,用于密钥管理。

    则Alice向Bob通信时:

    Alice:

    \begin{enumerate}
        \item 对原文$msg$做哈希,得到消息摘要:$h = h(SK_A, msg)$;
        \item 对消息摘要$h$用Alice的私钥$SK_A$做数字签名,得到签名:\\
        $s = SIGN_{RSA}(SK_A, h)$;
        \item 生成随机密钥$k_A$,对原文和数字签名进行加密,得到密文:\\
        $c_1 = ENC_{AES}(k_A, msg||s)$;
        \item 通过CA获得Bob的公钥$PK_B$,对$k_A$加密,得到数字信封:\\
        $c_2 = ENC_{RSA}(PK_B, k_A)$;
        \item 将$c_1,c_2$传输给Bob。
    \end{enumerate}
    
    Bob:
    \begin{enumerate}
        \item 将$c_2$用Bob的私钥$SK_B$解密,得到:\\
        $k_A = DEC_{RSA}(SK_B, c_2)$
        \item 将$c_1$用$k_A$解密,得到:$msg||s = DEC_{AES}(k_A, c_1)$
        \item 通过CA获得Alice的公钥$PK_A$,对$s$验签,得到传输消息摘要:\\
        $h = VERIFY_{RSA}(PK_A, s)$
        \item 对原文$msg$做哈希,得到消息摘要:$h' = h(SK_A, msg)$;
        \item 比对$h$和$h'$。
    \end{enumerate}

    此时,通信的安全性验证:
    \begin{enumerate}
        \item 若Bob在第1步能用自己的私钥$SK_B$解密,则说明该信息是发给Bob的;
        \item 若Bob在第3步能用Alice的公钥$SK_A$验签,则说明该信息的发送者是Alice;
        \item 若Bob在第5步对消息摘要比对成功,则说明原文没有被篡改。
    \end{enumerate}

    从而,我们可将该通信方法的安全性规约到RSA和AES的安全性上。

    \newpage
    (2) 该方案能抵御中间人攻击。

    此处Alice和Bob都是通过CA获得对方的公钥信息。而CA会存储提交申请的公钥信息,并对用CA的公钥$SK_{CA}$做数字签名。
    因此,Alice和Bob都可以通过CA的公钥$PK_{CA}$来验证对方的公钥信息。此处保证从CA签发的证书不可篡改。

    但我们仍需考虑是否在Alice/Bob向CA提交申请时,有中间人篡改Alice, Bob的公钥。若该传输发生在安全信道上,
    则无需考虑;若发生在不安全信道上,需要Alice在申请通过后向CA询问自己的公钥,由于CA->Alice传输过程
    是安全的,因此Alice能够判断自己的申请信息是否被篡改,从而重新申请。

    因此,该方案能够抵御中间人攻击。
    
\end{Solution}

\section{AES完整性攻击}

假设 Alice 需要向 Bob 支付 100 元,她使用随机向量 IV 下的 CBC 模
式进行 AES 分组加密,得到如下密文:

\begin{center}
    20814804c1767293b99f1d9cab3bc3e7 ac1e37bfb15599e5f40eef805488281d
\end{center}
已知上述密文对应的明文为:"Pay Bob 100\$"。请对上述密文进行修改,得
到"Pay Bob 500\$" 对应的密文。请给出计算方法和过程。

\begin{Solution}
    由于采用CBC模式,因此前16个字节为IV,即:
    \begin{equation}
        IV = \text{20814804c1767293b99f1d9cab3bc3e7}
    \end{equation}

        如下图所示,为CBC模式的揭秘过程,对应$P_0=IV\oplus C_0$,因此仅需修改$IV$,就可以使得$P_0$解密错误。
    \begin{figure}[!htbp]
        \centering
        \includegraphics[width=0.6\textwidth]{figures/cbc_decryption.png}
    \end{figure}
    
    查表得$ord('1') = 49, ord('5') = 53, ord('1') \oplus ord('5') = 0\text{x}04$,设置覆盖串为:

    $s = \text{0x00000000000000000400000000000000}$

    $IV' = IV \oplus s =~$0x20814804c1767293b\underline{d}9f1d9cab3bc3e7

    因此最终密文修改为:
    \begin{center}
        20814804c1767293b\underline{d}9f1d9cab3bc3e7 ac1e37bfb15599e5f40eef805488281d
    \end{center}
\end{Solution}

\newpage
\chapter{第五次作业}

\section{基于ElGamal的代理重加密}

Bob 所在公司的邮件服务器发布了一个公钥$PK_{Bob}$,这样所有 Bob 收
到的电子邮件可以在$PK_{Bob}$下加密以保证机密性。当 Bob 去度假时,他指
示公司的邮件服务器将他收到的所有电子邮件转发给他的同事 Alice。Alice
的公钥是$PK_{Alice}$。邮件服务器需要一种方法将公钥 $PK_{Bob}$下加密的电
子邮件“翻译”成使用公钥$PK_{Alice}$加密的电子邮件。如果邮件服务器有
$SK_{Bob}$,“翻译”将容易完成,但随后邮件服务器可以读取 Bob 收到的所有
电子邮件,所以是不可取的。因此基于 ElGamal 加密方案进行修改,得到如
下方案(代理重加密):

令$\mathbb{G}$是阶为素数$q$的循环群,$g\in \mathbb{G}$是生成元。当使用公钥$PK=u=g^\alpha\in\mathbb{G}$
进行加密时,其过程如下:

\begin{itemize}
    \item 随机选取$\beta\stackrel{R}{\leftarrow} \mathbb{Z}_p$, 计算$v=g^\beta, k=H(u^\beta)$,
    \item 计算$c=E_s(k,m)$,输出$(v,c)$作为密文。
\end{itemize}

其中$E_s$是对称加密算法,密钥空间为$\mathcal{K}_s$,$H:\mathbb{G}\rightarrow\mathcal{K}_s$是一个哈希函数。

令Alice的私钥为$\alpha \in \mathbb{Z}_q$,Bob的私钥为$\alpha' \in \mathbb{Z}_q$。为了实现从公钥
$PK_{Bob}$下密文到$PK_{Alice}$下密文的隐私保护“翻译”,Alice和Bob共同计算$\tau = \alpha / \alpha'$,并将其发送给邮件服务器。

\begin{enumerate}
    \item 请解释邮件服务器如何用$\tau$将密文$c=E(PK_{Bob},m)$“翻译”成Alice可以解密的密文$c'$;
    \item 请分析:如果邮件服务器可以使用$\tau$解密发送给Bob的邮件,那么他也可以攻破修改后ElGamal方案的安全性。
    即,证明若存在一个高效的算法$\mathcal{A}$,给定$(g,g^\alpha, g^{\alpha'},\tau, c=E(PK_{Bob},m))$作为输入,
    能够输出$m$,那么存在一个算法$\mathcal{B}$能够利用算法$\mathcal{A}$攻破ElGamal的语义安全性。
\end{enumerate}

\begin{Solution}
    (1) 根据所给的加密方案 $c = (v_0, c_0)$,其中$v_0=g^{\beta}$是一个随机数。

    则此时邮件服务器需要完成的重加密过程为:
    \begin{enumerate}
        \item 计算 $v_1 = v_0 ^ \tau$;
        \item 将 $c' = (v_1, c_0)$发送给Alice。
    \end{enumerate}

    Alice在收到重加密密文$(v_1, c_0)$后,可计算:

    \begin{enumerate}
        \item $v_1^{\alpha'} = v_0 ^ {\alpha'\tau} = g^{\alpha'\beta \alpha / \alpha' }
        =g^{\alpha\beta}$;
        \item $k = H(g^{\alpha\beta}) = H({PK_{Bob}}^\beta)$;
        \item $m = D_s(k, c_0)$,其中$D_s$为对称加密算法$E_s$对应的解密算法。
    \end{enumerate}
    
    (2) 如下图\ref{fig:1}所示,ElGamal密码体系语义安全的示意图。
    \begin{figure}[!htbp]
        \centering
        \includegraphics[width=0.4\textwidth]{figures/1.png}
        \caption{ElGamal密码体系语义安全}
        \label{fig:1}
    \end{figure}

    构造算法$\mathcal{B}=\mathcal{A}(g,g^a,g^a,\tau = a/a = 1,c)$,敌手给出不同消息$m_0,m_1$后,
    可根据算法$\mathcal{B}$对挑战者返回的密文进行解密。
    
    则此时敌手的优势:
    \begin{equation}
        \begin{aligned}
            \mathrm{Adv_{ElGamal}}[A, E_{\mathrm{ElGamal}}]    &= |\mathrm{Pr}[EXP(0)=1] - \mathrm{Pr}[EXP(1)=1]|\\
                                        &= 1
        \end{aligned}
    \end{equation}

    即攻破了ElGamal的语义安全性。

\end{Solution}


\section{RSA 共用模数下的安全性分析}

在基于RSA的陷门置换中,每个用户需要使用不同的模数$N=pq$。假设多个用户使用同一个模数
$N=pq$,每个用户使用不同的公钥$e_i\in \mathbb{Z}$和私钥$d_i \in \mathbb{Z}$,
其中$e_i\cdot d_i \equiv 1 \bmod \varphi(N)$。从表面上看,似乎该方案是可行的。需要签署消息
$m \in \mathcal{M}$时,Alice计算$\sigma_A=H(m)^{d_a}$,其中$H:\mathcal{M}\rightarrow \mathbb{Z}_N^*$
是一个哈希函数。类似的当Bob需要签署消息时,计算$\sigma_b = H(m)^{d_b}$。由于只有Alice知道$d_a$的值,
只有Bob知道$d_b$的值,所以方案看起来似乎有合理性。然而方法是存在安全性攻击的:Bob可以使用他的私钥伪造
Alice的签名。

\begin{enumerate}
    \item 请分析Bob可以使用其公私钥对$(e_b,d_b)$获取任何一个$\varphi(N)$的倍数$v$。
    \item 如果Bob知道Alice的公钥$e_a$,对于任意的消息$m$,请尝试计算$\sigma = H(m)^{1/e_a}$,
    进而完成签名的伪造。
\end{enumerate}

\begin{Solution}
    (1) 由同余方程$e_i\cdot d_i \equiv 1 \bmod \varphi(N)$可得:
    \begin{equation}
        e_b\cdot d_b = 1 + k\varphi(N),~k\in\mathbb{N}
    \end{equation}
    由Euler定理,对$a\in \mathbb{Z}_N$,若$\gcd(a, N) = 1$,则$a^{\varphi(N)} \equiv 1 \bmod(N)$,则有:
    \begin{equation}
        a^{k\varphi(N)} \equiv a^{e_b\cdot d_b-1} \equiv 1 \bmod(N)
    \end{equation}
    又由于$e_b\cdot d_b-1$为偶数,设$e_b\cdot d_b-1 = 2^st$,其中$t$是一个奇数,则有
    $ a^{2^st} \equiv 1 \bmod(N)$。

    在该同余方程下进行攻击:
    \begin{enumerate}
        \item 取随机数$a \in \mathbb{Z}_N^*$,此时$\gcd(a, N) = 1$恒成立;
        \item 计算$k = e_b\cdot d_b - 1$;
        \item 计算$x = a^{k/2},a^{k/4},\cdots,a^{k/2^t} (\bmod N)$,直至$x>1$且$y=\gcd(x-1, N)>1$
        \item 若无这样的$y$,重新取随机数。
    \end{enumerate}

    事实上,上述攻击流程中得到的$y$,恰为$N$的一个因子$p$,根据扩展欧几里得算法易求得$q=p^{-1} \bmod N$,此时即求得$\varphi(N)$。
    从而即可求得$\varphi(N)$的任意倍数$v$。

    注:在上述攻击中,可证明对于至少一半的$a\in\mathbb{Z}_N^*$,存在$i\in[1,s]$使得:
    \begin{equation}
        a^{2^{i-1}t} \not\equiv \pm 1 (\bmod N), a^{2^{i}t} \equiv 1(\bmod N)
    \end{equation}
    若满足,则$\gcd(a^{2^{i-1}t}-1, N)$是$N$的一个非平凡因子\footnote{https://ctf-wiki.org/crypto/asymmetric/rsa/d\_attacks/rsa\_d\_attack/}。

    (2) 由于Alice和Bob的公私钥共模数,且Bob已被证明可对$N$进行分解。因此此时可以直接计算出Alice的私钥:
    \begin{equation}
        d_a = e_a^{-1} \bmod \varphi(N)
    \end{equation}

    于是便可对任意消息伪造Alice的签名$\sigma = H(m)^{1/e_a} = \sigma = H(m)^{d_a}$
\end{Solution}