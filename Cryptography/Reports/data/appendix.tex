\chapter{实验代码}

\section{Shamir 秘密共享}
\label{appendix:Shamir}

shamir.py

\begin{lstlisting}[language = Python]
from Crypto.Util.py3compat import is_native_int
from Crypto.Util import number
from Crypto.Util.number import long_to_bytes, bytes_to_long
from Crypto.Random import get_random_bytes as rng


def _mult_gf2(f1, f2):
    """Multiply two polynomials in GF(2)"""

    # Ensure f2 is the smallest
    if f2 > f1:
        f1, f2 = f2, f1
    z = 0
    while f2:
        if f2 & 1:
            z ^= f1
        f1 <<= 1
        f2 >>= 1
    return z


def _div_gf2(a, b):
    """
    Compute division of polynomials over GF(2).
    Given a and b, it finds two polynomials q and r such that:

    a = b*q + r with deg(r)<deg(b)
    """

    if (a < b):
        return 0, a

    deg = number.size
    q = 0
    r = a
    d = deg(b)
    while deg(r) >= d:
        s = 1 << (deg(r) - d)
        q ^= s
        r ^= _mult_gf2(b, s)
    return (q, r)


class _Element(object):
    """Element of GF(2^128) field"""

    # The irreducible polynomial defining this field is 1+x+x^2+x^7+x^128
    irr_poly = 1 + 2 + 4 + 128 + 2 ** 128

    def __init__(self, encoded_value):
        """Initialize the element to a certain value.

        The value passed as parameter is internally encoded as
        a 128-bit integer, where each bit represents a polynomial
        coefficient. The LSB is the constant coefficient.
        """

        if is_native_int(encoded_value):
            self._value = encoded_value
        # elif len(encoded_value) == 16:
        else:
            self._value = bytes_to_long(encoded_value)
        # else:
        #     raise ValueError("The encoded value must be an integer or a 16 byte string")

    def __eq__(self, other):
        return self._value == other._value

    def __int__(self):
        """Return the field element, encoded as a 128-bit integer."""
        return self._value

    def encode(self):
        """Return the field element, encoded as a 16 byte string."""
        return long_to_bytes(self._value)

    def __mul__(self, factor):

        f1 = self._value
        f2 = factor._value

        # Make sure that f2 is the smallest, to speed up the loop
        if f2 > f1:
            f1, f2 = f2, f1

        if self.irr_poly in (f1, f2):
            return _Element(0)

        mask1 = 2 ** 128
        v, z = f1, 0
        while f2:
            # if f2 ^ 1: z ^= v
            mask2 = int(bin(f2 & 1)[2:] * 128, base=2)
            z = (mask2 & (z ^ v)) | ((mask1 - mask2 - 1) & z)
            v <<= 1
            # if v & mask1: v ^= self.irr_poly
            mask3 = int(bin((v >> 128) & 1)[2:] * 128, base=2)
            v = (mask3 & (v ^ self.irr_poly)) | ((mask1 - mask3 - 1) & v)
            f2 >>= 1
        return _Element(z)

    def __add__(self, term):
        return _Element(self._value ^ term._value)

    def inverse(self):
        """Return the inverse of this element in GF(2^128)."""

        # We use the Extended GCD algorithm
        # http://en.wikipedia.org/wiki/Polynomial_greatest_common_divisor

        if self._value == 0:
            raise ValueError("Inversion of zero")

        r0, r1 = self._value, self.irr_poly
        s0, s1 = 1, 0
        while r1 > 0:
            q = _div_gf2(r0, r1)[0]
            r0, r1 = r1, r0 ^ _mult_gf2(q, r1)
            s0, s1 = s1, s0 ^ _mult_gf2(q, s1)
        return _Element(s0)

    def __pow__(self, exponent):
        result = _Element(self._value)
        for _ in range(exponent - 1):
            result = result * self
        return result


class Shamir(object):
    """Shamir's secret sharing scheme.

    A secret is split into ``n`` shares, and it is sufficient to collect
    ``k`` of them to reconstruct the secret.
    """

    @staticmethod
    def split(k, n, secret):
        """Split a secret into ``n`` shares.

        The secret can be reconstructed later using just ``k`` shares
        out of the original ``n``.
        Each share must be kept confidential to the person it was
        assigned to.

        Each share is associated to an index (starting from 1).

        Args:
          k (integer):
            The sufficient number of shares to reconstruct the secret (``k < n``).
          n (integer):
            The number of shares that this method will create.
          secret (byte string):
            A byte string of 16 bytes (e.g. the AES 128 key).

        Return (tuples):
            ``n`` tuples. A tuple is meant for each participant and it contains two items:

            1. the unique index (an integer)
            2. the share (a byte string, 16 bytes)
        """

        #
        # We create a polynomial with random coefficients in GF(2^128):
        #
        # p(x) = \sum_{i=0}^{k-1} c_i * x^i
        #
        # c_0 is the encoded secret
        #

        coeffs = [_Element(rng(16)) for i in range(k - 1)]
        coeffs.append(_Element(secret))

        # Each share is y_i = p(x_i) where x_i is the public index
        # associated to each of the n users.

        def make_share(user, coeffs):
            idx = _Element(user)
            share = _Element(0)
            for coeff in coeffs:
                share = idx * share + coeff

            return share.encode()

        return [(i, make_share(i, coeffs)) for i in range(1, n + 1)]

    @staticmethod
    def combine(shares):
        """Recombine a secret, if enough shares are presented.

        Args:
          shares (tuples):
            The *k* tuples, each containin the index (an integer) and
            the share (a byte string, 16 bytes long) that were assigned to
            a participant.
          ssss (bool):
            If ``True``, the shares were produced by the ``ssss`` utility.
            Default: ``False``.

        Return:
            The original secret, as a byte string (16 bytes long).
        """

        k = len(shares)

        gf_shares = []
        for x in shares:
            idx = _Element(x[0])
            value = _Element(x[1])
            if any(y[0] == idx for y in gf_shares):
                raise ValueError("Duplicate share")
            gf_shares.append((idx, value))

        result = _Element(0)
        for j in range(k):
            x_j, y_j = gf_shares[j]

            numerator = _Element(1)
            denominator = _Element(1)

            for m in range(k):
                x_m = gf_shares[m][0]
                if m != j:
                    numerator *= x_m
                    denominator *= x_j + x_m
            result += y_j * numerator * denominator.inverse()
        return result.encode()
\end{lstlisting}

\newpage
widget.py
\begin{lstlisting}[language = Python]
# This Python file uses the following encoding: utf-8
import os
from pathlib import Path
import sys

from shamir import Shamir
from binascii import hexlify, unhexlify

from PySide2.QtWidgets import QApplication, QWidget
from PySide2.QtCore import QFile
from PySide2.QtUiTools import QUiLoader


class Widget(QWidget):
    def __init__(self):
        super(Widget, self).__init__()
        self.ui = QUiLoader().load('form.ui')
        self.ui.show()
        self.ui.pushButton_create_share.clicked.connect(self.create_shares)
        self.ui.pushButton_combine_share.clicked.connect(self.combine_shares)

    def create_shares(self):
        secret = self.ui.plainTextEdit_Secret.toPlainText().encode()
        l = Shamir.split(3, 5, secret)
        print(l)
        self.ui.plainTextEdit_s1.setPlainText(str(hexlify(l[0][1]))[2:-1])
        self.ui.plainTextEdit_s2.setPlainText(str(hexlify(l[1][1]))[2:-1])
        self.ui.plainTextEdit_s3.setPlainText(str(hexlify(l[2][1]))[2:-1])
        self.ui.plainTextEdit_s4.setPlainText(str(hexlify(l[3][1]))[2:-1])
        self.ui.plainTextEdit_s5.setPlainText(str(hexlify(l[4][1]))[2:-1])
        pass

    def combine_shares(self):
        self.ui.plainTextEdit_cs.setPlainText("NULL")
        shares = []
        if len(self.ui.plainTextEdit_s1.toPlainText()) != 0:
            shares.append((1, unhexlify(self.ui.plainTextEdit_s1.toPlainText().encode())))
        if len(self.ui.plainTextEdit_s2.toPlainText()) != 0:
            shares.append((2, unhexlify(self.ui.plainTextEdit_s2.toPlainText().encode())))
        if len(self.ui.plainTextEdit_s3.toPlainText()) != 0:
            shares.append((3, unhexlify(self.ui.plainTextEdit_s3.toPlainText().encode())))
        if len(self.ui.plainTextEdit_s4.toPlainText()) != 0:
            shares.append((4, unhexlify(self.ui.plainTextEdit_s4.toPlainText().encode())))
        if len(self.ui.plainTextEdit_s5.toPlainText()) != 0:
            shares.append((5, unhexlify(self.ui.plainTextEdit_s5.toPlainText().encode())))
        print(shares)
        secret = Shamir.combine(shares)
        self.ui.plainTextEdit_cs.setPlainText(secret.decode())
        pass


if __name__ == "__main__":
    app = QApplication([])
    widget = Widget()
    widget.show()
    sys.exit(app.exec_())
\end{lstlisting}

\newpage
form.ui
\begin{lstlisting}[language = html]
<?xml version="1.0" encoding="UTF-8"?>
<ui version="4.0">
 <class>Widget</class>
 <widget class="QWidget" name="Widget">
  <property name="geometry">
   <rect>
    <x>0</x>
    <y>0</y>
    <width>557</width>
    <height>313</height>
   </rect>
  </property>
  <property name="windowTitle">
   <string>Widget</string>
  </property>
  <layout class="QGridLayout" name="gridLayout">
   <item row="0" column="0" colspan="3">
    <widget class="QLabel" name="label">
     <property name="text">
      <string>3, 5 - Shamir Secret Sharing</string>
     </property>
    </widget>
   </item>
   <item row="1" column="0">
    <widget class="QLabel" name="label_2">
     <property name="text">
      <string>Secret</string>
     </property>
    </widget>
   </item>
   <item row="1" column="1">
    <widget class="QLabel" name="label_3">
     <property name="text">
      <string>Share 1</string>
     </property>
    </widget>
   </item>
   <item row="1" column="2">
    <widget class="QLabel" name="label_4">
     <property name="text">
      <string>Share 2</string>
     </property>
    </widget>
   </item>
   <item row="1" column="3">
    <widget class="QLabel" name="label_8">
     <property name="text">
      <string>Combined Share</string>
     </property>
    </widget>
   </item>
   <item row="2" column="0">
    <widget class="QPlainTextEdit" name="plainTextEdit_Secret"/>
   </item>
   <item row="2" column="1">
    <widget class="QPlainTextEdit" name="plainTextEdit_s1"/>
   </item>
   <item row="2" column="2">
    <widget class="QPlainTextEdit" name="plainTextEdit_s2"/>
   </item>
   <item row="2" column="3" rowspan="3">
    <widget class="QPlainTextEdit" name="plainTextEdit_cs"/>
   </item>
   <item row="3" column="0">
    <widget class="QLabel" name="label_5">
     <property name="text">
      <string>Share 3</string>
     </property>
    </widget>
   </item>
   <item row="3" column="1">
    <widget class="QLabel" name="label_6">
     <property name="text">
      <string>Share 4</string>
     </property>
    </widget>
   </item>
   <item row="3" column="2">
    <widget class="QLabel" name="label_7">
     <property name="text">
      <string>Share 5</string>
     </property>
    </widget>
   </item>
   <item row="4" column="0">
    <widget class="QPlainTextEdit" name="plainTextEdit_s3"/>
   </item>
   <item row="4" column="1">
    <widget class="QPlainTextEdit" name="plainTextEdit_s4"/>
   </item>
   <item row="4" column="2">
    <widget class="QPlainTextEdit" name="plainTextEdit_s5"/>
   </item>
   <item row="5" column="0">
    <widget class="QPushButton" name="pushButton_create_share">
     <property name="text">
      <string>Create Shares</string>
     </property>
    </widget>
   </item>
   <item row="5" column="1">
    <widget class="QPushButton" name="pushButton_combine_share">
     <property name="text">
      <string>Combine Shares</string>
     </property>
    </widget>
   </item>
  </layout>
 </widget>
 <resources/>
 <connections/>
</ui>

\end{lstlisting}

\newpage
image.py
\begin{lstlisting}[language = Python]
from statistics import mode
import numpy as np
import matplotlib.pyplot as plt
from shamir import *
from binascii import hexlify

img = plt.imread('cat.jpg')
# plt.imshow(img)

def img_split(k, n, img: np.ndarray):
    a, b, c = img.shape
    l = []
    for i in range(n):
        l.append(np.zeros((a, b, c), np.dtype('int')))
    for i in range(a):
        for j in range(b):
            for kk in range(c):
                tmp_share = Shamir.split(k, n, int(img[i, j, kk]))
                for idx in range(n):
                    l[idx][i, j, kk] = bytes_to_long(tmp_share[idx][1])
    return l

tmp_l = img_split(3, 5, img)
tmp_l.append(img)

print('imhere')

def show_images(images: list[np.ndarray]) -> None:
    n: int = len(images)
    f = plt.figure()
    for i in range(n):
        # Debug, plot figure
        f.add_subplot(1, n, i + 1)
        plt.imshow(images[i])

    plt.show(block=True)

show_images(tmp_l)

plt.show()
\end{lstlisting}

\newpage
\section{Many Time Pad}
\label{appendix:mtp}
main.py
\begin{lstlisting}[language = Python]
from binascii import unhexlify
from string import ascii_letters
from math import sqrt
import utils

def main():
    cipher_texts = []
    with open('cipher_text.txt', encoding='utf-8') as file:
        content = file.read().split('\n')
    for i in range(1, 21, 2):
        cipher_texts.append(unhexlify(content[i]))

    with open('dest_text.txt', encoding='utf-8') as file:
        content = file.read()
    dest_text = unhexlify(content)

    letters = ascii_letters.encode('ascii')     ## ASCII letters

    key = [0] * 1024

    for ct_a in cipher_texts:
        possible_pos = [0] * len(ct_a)
        for ct_b in cipher_texts:
            if ct_a == ct_b:
                continue
            guess_str = utils.byte_xor(ct_a, ct_b)
            for idx, guess_char in enumerate(guess_str):
                if guess_char not in letters and guess_char != 0:
                    continue
                possible_pos[idx] += 1

        threshold = len(cipher_texts) - sqrt(len(cipher_texts))   ## Magic threshold

        for i in range(len(ct_a)):
            if possible_pos[i] > threshold:
                key[i] = ct_a[i] ^ 0x20

    print(utils.byte_xor(dest_text, key))


if __name__ == '__main__':
    main()
\end{lstlisting}

\newpage
utils.py
\begin{lstlisting}[language = Python]
def byte_xor(a: bytes, b: bytes) -> bytes:
    '''
    :return: xor result of a and b
    '''
    if len(a) > len(b):
        return bytes([x ^ y for x, y in zip(a[:len(b)], b)])
    else:
        return bytes([x ^ y for x, y in zip(a, b[:len(a)])])
\end{lstlisting}


\newpage
\section{AES}
\label{appendix:aes}
main.py
\begin{lstlisting}[language = Python]
from binascii import unhexlify
from utils import *

def main():
    keys = []
    cipher_texts = []
    with open('infos.txt', encoding='utf-8') as file:
        content = file.read().split('\n')
    for i in range(1, 16, 4):
        keys.append(unhexlify(content[i]))
        cipher_texts.append(unhexlify(content[i+2]))

    print(CBC.decrypt(cipher_texts[0], keys[0]))
    print(CBC.decrypt(cipher_texts[1], keys[1]))
    print(CTR.decrypt(cipher_texts[2], keys[2]))
    print(CTR.decrypt(cipher_texts[3], keys[3]))

if __name__ == '__main__':
    main()
\end{lstlisting}

utils.py
\begin{lstlisting}[language = Python]
import binascii
from Crypto.Cipher import AES
from cxc_toolkit import integer

def byte_xor(a: bytes, b: bytes) -> bytes:
    '''
    :return: xor result of a and b
    '''
    if len(a) > len(b):
        return bytes([x ^ y for x, y in zip(a[:len(b)], b)])
    else:
        return bytes([x ^ y for x, y in zip(a, b[:len(a)])])

def to_int(byte):
    """
    Convert bytes to int

    :type byte: bytes
    :rtype: int
    """
    s = 0
    for i, number in enumerate(byte):
        s = s * 256 + number
    return s


def byte_add(byte, addtions):
    """
    Add int to bytes

    :type byte: bytes
    :type addtions: int
    :rtype: bytes
    """
    return integer.to_bytes(to_int(byte) + addtions, bytes_size=len(byte))

def msg_block_generator(msg, padding=False):
        while len(msg) >= 16:
            yield msg[:16]
            msg = msg[16:]
        if len(msg) > 0:
            if not padding:
                yield msg
                return

            reminder = 16 - len(msg)
            msg = msg + bytes([reminder]) * reminder
            yield msg
        else:
            yield b'16' * 16


def cipher_block_generator(cipher):
    while len(cipher):
        yield cipher[:16]
        cipher = cipher[16:]

class CBC:

    def encrypt(msg, key, iv):
        cipher = AES.new(key, AES.MODE_ECB)
        cipher_block = iv
        ciphertext = iv
        for msg_block in msg_block_generator(msg, padding=True):
            cipher_block = cipher.encrypt(byte_xor(cipher_block, msg_block))
            ciphertext += cipher_block
        return ciphertext


    def decrypt(cipher_text, key):
        cipher = AES.new(key, AES.MODE_ECB)
        iv, cipher_text = cipher_text[:16], cipher_text[16:]
        msg = b''
        for cipher_block in cipher_block_generator(cipher_text):
            msg_block = byte_xor(cipher.decrypt(cipher_block), iv)
            iv = cipher_block
            msg += msg_block
        if msg[-16:] == b'\x16' * 16:
            return msg[:-16]
        pad_bytes = msg[-1]
        reminder = len(msg) - pad_bytes
        if msg[reminder:] == bytes([pad_bytes]) * pad_bytes:
            return msg[:reminder]
        else:
            print('Cipher text is invalid')


class CTR:

    def encrypt(msg, key, iv):
        cipher = AES.new(key, AES.MODE_ECB)
        ciphertext = b''
        for i, msg_block in enumerate(msg_block_generator(msg, padding=False)):
            cipher_block = cipher.encrypt(byte_add(iv, i))
            cipher_block = byte_xor(msg_block, cipher_block)
            ciphertext += cipher_block
        return ciphertext

    def decrypt(cipher_text, key):
        iv, cipher_text = cipher_text[:16], cipher_text[16:]
        cipher = AES.new(key, AES.MODE_ECB)
        msg = b''
        for i, cipher_block in enumerate(cipher_block_generator(cipher_text)):
            iv_encrypted = cipher.encrypt(byte_add(iv, i))
            msg_block = byte_xor(cipher_block, iv_encrypted)
            msg += msg_block
        return msg

\end{lstlisting}