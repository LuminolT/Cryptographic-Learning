\chapter{第一周作业}

\newpage
\chapter{第二周作业}

\section{第3题}

\textbf{题目}:考察$\mathbb{Z}_2[x]$中的多项式$f_1(x)=x^3+x+1$和$f_2(x)=x^3+1$,


(1) 判断$f_1(x),f_2(x)$哪个是$\mathbb{Z}_2[x]$上的不可约多项式;

\begin{Solution}
    $\mathbb{Z}_2[x]$中的一次因式有:$x,x-1$,则有:
    \begin{equation}
        \begin{aligned}
            f_1(0) = 1, f_1(1) = 1\\
            f_2(0) = 1, f_2(1) = 0\\
        \end{aligned}
    \end{equation}
    
    由余元定理的推论可知:$(x-1)|f_2(x)$,而$f_1(x)$无$\mathbb{Z}_2[x]$上的一次因式。
    
    $\therefore f_1(x)$为$\mathbb{Z}_2[x]$上的不可约多项式,$f_2(x)$为$\mathbb{Z}_2[x]$上的可约多项式。
\end{Solution}

(2) 将找到的不可约多项式记为$f(x)$,写出$\mathbb{Z}_2[x]$关于模$f(x)$的剩余类$\mathbb{Z}_2[x]_{f(x)}$的全部元素;

\begin{Solution}
    由(1),$f(x)=f_1(x)=x^3+x+1$,$\partial^o(f(x))=3$
    
    因此剩余类中多项式$r(x)$满足$0\le \partial^o(r(x)) < 3$,因此模$f(x)$剩余类为:
    \begin{equation}
        \mathbb{Z}_2[x]_{f(x)} = \left\{ax^2+bx+c|a,b,c \in \{0,1\}\right\}
    \end{equation}
\end{Solution}

(3) 在集合$\mathbb{Z}_2[x]_{f(x)}$上定义加法运算$\oplus$和乘法运算$\otimes$分别为:
\begin{equation}
    \begin{aligned}
        a(x) \oplus b(x) = a(x) + b(x) \\
        a(x) \otimes b(x) = (a(x)b(x))_{f(x)}
    \end{aligned}
\end{equation}

请求出$(x+1)\oplus(x^2+x+1)$和$(x+1)\otimes(x^2+x+1)$的值;

\begin{Solution}
    根据给定的运算,得到:
    \begin{equation}
        \begin{aligned}
            (x+1)\oplus(x^2+x+1)    &= (x+1)+(x^2+x+1)\\
                                    &= x^2 + 2x + 2\\
                                    &= x^2
        \end{aligned}
    \end{equation}
    \begin{equation}
        \begin{aligned}
        (x+1)\otimes(x^2+x+1)   &= ((x+1)(x^2+x+1)) \bmod f(x)\\
                                &= (x^3+x^2+x+x^2+x+1) \bmod f(x)\\
                                &= (x^3+2x^2+2x+1) \bmod f(x)\\
                                &= (x^3 +2x^2+ 1) \bmod (f(x))\\
                                &= x
        \end{aligned}
    \end{equation}
\end{Solution}

(4) 求出$x+1$关于乘法运算$\otimes$的逆元。

\begin{Solution}
    用欧几里得算法计算$f(x)$和$x+1$的最大公因式:
    \begin{equation}
        \begin{aligned}
            f(x) &= (x^2 + x)(x+1) + 1\\
            x+1 &= (x+1) * 1 + 0\\
        \end{aligned}
    \end{equation}

    $\therefore gcd(f(x), x+1) = x+1$,进一步得到$1=f(x)+(x^2+x)(x+1)$

    $\therefore (x+1)(x^2+x) \equiv 1 (\bmod f(x))$

    $\therefore (x+1)^{-1} \equiv x^2+x (\bmod f(x))$

    即$x+1$关于乘法运算$\otimes$的逆元是$x^2+x$.
\end{Solution}

\newpage
\chapter{第三周作业}

\section{第4题}

\textbf{题目}:设$p=7$,用$\mathbb{Z}_7$来记所有整数集合$\mathbb{Z}$模$7$的等价类的集合,在$\mathbb{Z}_7$上定义加法运算$\oplus$
和乘法运算分别为:
\begin{equation}
    \begin{aligned}
        a \oplus b= (a+b)_7\\
        a \otimes b = (ab)_7\\
    \end{aligned}
\end{equation}

(1) 写出模$7$等价类的集合$\mathbb{Z}_7$的全部元素;

\begin{Solution}
    $\mathbb{Z}_7 = \left\{0,1,2,3,4,5,6\right\}$
\end{Solution}

(2) 请列出$\mathbb{Z}_7$关于$\oplus,\otimes$这两种运算的运算表;

\begin{Solution}
    如下表所示:

    \begin{minipage}[t]{0.45\linewidth}
            \centering
            \makeatletter\def\@captype{table}\makeatother\caption{$\mathbb{Z}_7$上$\oplus$运算表}
            \begin{tabular}{|c|c|c|c|c|c|c|c|}
            \hline
            $\oplus$ & 0 & 1 & 2 & 3 & 4 & 5 & 6 \\ \hline
            0        & 0 & 1 & 2 & 3 & 4 & 5 & 6 \\ \hline
            1        & 1 & 2 & 3 & 4 & 5 & 6 & 0 \\ \hline
            2        & 2 & 3 & 4 & 5 & 6 & 0 & 1 \\ \hline
            3        & 3 & 4 & 5 & 6 & 0 & 1 & 2 \\ \hline
            4        & 4 & 5 & 6 & 0 & 1 & 2 & 3 \\ \hline
            5        & 5 & 6 & 0 & 1 & 2 & 3 & 4 \\ \hline
            6        & 6 & 0 & 1 & 2 & 3 & 4 & 5 \\ \hline
            \end{tabular}
    \end{minipage}%
    \begin{minipage}[t]{0.45\linewidth}
            \centering
            \makeatletter\def\@captype{table}\makeatother\caption{$\mathbb{Z}_7$上$\otimes$运算表}
            \begin{tabular}{|c|c|c|c|c|c|c|c|}
            \hline
            $\otimes$ & 0 & 1 & 2 & 3 & 4 & 5 & 6 \\ \hline
            0        & 0 & 0 & 0 & 0 & 0 & 0 & 0 \\ \hline
            1        & 0 & 1 & 2 & 3 & 4 & 5 & 6 \\ \hline
            2        & 0 & 2 & 4 & 6 & 1 & 3 & 5 \\ \hline
            3        & 0 & 3 & 6 & 2 & 5 & 1 & 4 \\ \hline
            4        & 0 & 4 & 1 & 5 & 2 & 6 & 3 \\ \hline
            5        & 0 & 5 & 3 & 1 & 6 & 4 & 2 \\ \hline
            6        & 0 & 6 & 5 & 4 & 3 & 2 & 1 \\ \hline
            \end{tabular}
    \end{minipage}
    \\
\end{Solution}


    

(3) 试从域的定义出发验证$<\mathbb{Z}_7, \oplus, \otimes>$是一个有限域。

\begin{Solution}
    $<\mathbb{Z}_7, \oplus>$显然为Abel群,
    $7$是素数,因此$<\mathbb{Z}_7-\{0\}, \otimes>$也是Abel群,
    且对环$<\mathbb{Z}_7, \oplus, \otimes>$上元素$a,b,c$,满足:
    \begin{equation}
        \begin{aligned}
            a\otimes (b\oplus c)    & = (a ((b + c) \bmod 7)) \bmod 7 \\
                                    & = ((a \bmod 7)  ((b \bmod 7 + c \bmod 7) \bmod 7)) \bmod 7 \\
                                    & = (((a \bmod 7)(b \bmod 7) \bmod 7 + (a \bmod 7)(c \bmod 7) \bmod 7) \bmod 7 \\
                                    & = (((ab) \bmod 7)+((ac) \bmod 7)) \bmod 7\\
                                    & = ((a \otimes b) + (a \otimes c)) \bmod 7\\
                                    & = (a\otimes b) \oplus (a\otimes c)
        \end{aligned}
    \end{equation}

    即$<\mathbb{Z}_7, \oplus, \otimes>$中$\otimes$对$\oplus$满足左分配律,又乘法群为Abel群,因此可得:
    \begin{equation}
        (b\oplus c) \otimes a =  (b\otimes a) \oplus (c\otimes a)
    \end{equation}

    因此环$<\mathbb{Z}_7, \oplus, \otimes>$满足分配律,因此$<\mathbb{Z}_7, \oplus, \otimes>$是域。

    (事实上对素数$p$,同理可证明$<\mathbb{Z}_p, \oplus, \otimes>$为有限域)
\end{Solution}

\section{第5题}

用$\mathbb{Z}_8$来记所有整数集合$\mathbb{Z}$模$8$的等价类的集合,在$\mathbb{Z}_8$上定义加法运算$\oplus$
和乘法运算分别为:

\begin{equation}
    \begin{aligned}
        a \oplus b= (a+b)_8\\
        a \otimes b = (ab)_8\\
    \end{aligned}
\end{equation}

(1) 写出模$8$等价类的集合$\mathbb{Z}_8$的全部元素;

\begin{Solution}
    $\mathbb{Z}_8 = \left\{0,1,2,3,4,5,6,7\right\}$
\end{Solution}

(2) 请列出$\mathbb{Z}_8$关于$\oplus,\otimes$这两种运算的运算表;

\begin{Solution}
    如下表所示:

\begin{minipage}[t]{0.45\linewidth}
        \centering
        \makeatletter\def\@captype{table}\makeatother\caption{$\mathbb{Z}_8$上$\oplus$运算表}
        \begin{tabular}{|c|c|c|c|c|c|c|c|c|}
        \hline
        $\oplus$  & 0 &  1  & 2  & 3  & 4  & 5  & 6  & 7\\ \hline
        0  & 0 &  1  & 2  & 3  & 4  & 5  & 6  & 7\\ \hline
        1  & 1 &  2  & 3  & 4  & 5  & 6  & 7  & 0\\ \hline
        2  & 2 &  3  & 4  & 5  & 6  & 7  & 0  & 1\\ \hline
        3  & 3 &  4  & 5  & 6  & 7  & 0  & 1  & 2\\ \hline
        4  & 4 &  5  & 6  & 7  & 0  & 1  & 2  & 3\\ \hline
        5  & 5 &  6  & 7  & 0  & 1  & 2  & 3  & 4\\ \hline
        6  & 6 &  7  & 0  & 1  & 2  & 3  & 4  & 5\\ \hline
        7  & 7 &  0  & 1  & 2  & 3  & 4  & 5  & 6\\ \hline
        \end{tabular}
\end{minipage}%
\begin{minipage}[t]{0.45\linewidth}
        \centering
        \makeatletter\def\@captype{table}\makeatother\caption{$\mathbb{Z}_8$上$\otimes$运算表}
        \begin{tabular}{|c|c|c|c|c|c|c|c|c|}
        \hline
        $\otimes$  & 0  & 1  & 2 &  3 &  4 &  5 &  6 &  7\\ \hline
        0  & 0  & 0  & 0 &  0 &  0 &  0 &  0 &  0\\ \hline
        1  & 0  & 1  & 2 &  3 &  4 &  5 &  6 &  7\\ \hline
        2  & 0  & 2  & 4 &  6 &  0 &  2 &  4 &  6\\ \hline
        3  & 0  & 3  & 6 &  1 &  4 &  7 &  2 &  5\\ \hline
        4  & 0  & 4  & 0 &  4 &  0 &  4 &  0 &  4\\ \hline
        5  & 0  & 5  & 2 &  7 &  4 &  1 &  6 &  3\\ \hline
        6  & 0  & 6  & 4 &  2 &  0 &  6 &  4 &  2\\ \hline
        7  & 0  & 7  & 6 &  5 &  4 &  3 &  2 &  1\\ \hline
        \end{tabular}
\end{minipage}
\\
\end{Solution}

(3) 试从域的定义出发验证$<\mathbb{Z}_8, \oplus, \otimes>$不是一个有限域。

\begin{Solution}
    根据上述$\mathbb{Z}_8$上$\otimes$运算表发现,$2\otimes 4 =0$
因此显然代数系统$<\mathbb{Z}_8-\{0\}, \otimes>$不构成群,因此$<\mathbb{Z}_8, \oplus, \otimes>$不是一个有限域。
\end{Solution}

\newpage
\section{第6题}
比较第4题和第5题,想一想为什么第4题的7个元素不能构成有限域而第5题的8个元素能构成有限域。他们的运算有什么不同。

\begin{Solution}
    在第4题最后笔者给出了对素数$p$,同理可证明$<\mathbb{Z}_p, \oplus, \otimes>$为有限域。

    下面证明对于合数$n$,一定有$<\mathbb{Z}_n, \oplus, \otimes>$不是一个有限域。

    给定$n = p_1p_2\cdots p_m$,其中$p_i(1\le i \le m)$为素数。
    
    则取$p_1,\prod_{i=2}^mp_i\in <\mathbb{Z}_n - \{0\}, \otimes>$,显然有$p_1\otimes\prod_{i=2}^mp_i = n \bmod n = 0$

    而$0\notin <\mathbb{Z}_n - \{0\}, \otimes>$,该代数系统不封闭。
    
    因此$<\mathbb{Z}_n, \oplus, \otimes>$不是一个有限域。

    综上所述$<\mathbb{Z}_n, \oplus, \otimes>$是一个有限域当且仅当$n$为素数
\end{Solution}

\newpage
\chapter{第四周作业}
\section{第7题}
设$p=5$,考察$\mathbb{Z}_5[x]$中多项式
\begin{equation}
    p(x) = x^2 + 2
\end{equation}

(1) 试说明$p(x)$是$\mathbb{Z}_5[x]$上的不可约多项式;

\begin{Solution}
    $\mathbb{Z}_5[x]$中的一次因式有:$x,x-1,\cdots,x-4$,则有:
    \begin{equation}
        p(0) = 2, p(1) = 3, p(2) = 1, p(3) = 1, p(4) = 3
    \end{equation}
    
    由余元定理的推论可知:$p(x)$无$\mathbb{Z}_5[x]$上的一次因式。

    因此$p(x)$是$\mathbb{Z}_5[x]$上的不可约多项式。
\end{Solution}

(2) $\mathbb{Z}_5[x]_{p(x)}$是含有$5^2=25$个元素的集合,试写出$\mathbb{Z}_5[x]_{p(x)}$的全部元素;

\begin{Solution}
    $\partial^o(p(x))=2$
    ,则剩余类中多项式$r(x)$满足$0\le \partial^o(r(x)) < 2$,因此:
    \begin{equation}
        \begin{aligned}
            &\mathbb{Z}_5[x]_{f(x)} = \left\{ax+b|a,b \in \mathbb{Z}_5\right\}\\
            =&\{0,1,2,3,4,
                        x,x+1,x+2,x+3,x+4,
                        2x+1,2x+2,2x+3,2x+4,\\
                        &3x+1,3x+2,3x+3,3x+4,
                        4x+1,4x+2,4x+3,4x+4\}
        \end{aligned}
    \end{equation}
\end{Solution}

(3) 在$\mathbb{Z}_5[x]_{p(x)}$上定义加法运算$\oplus$和乘法运算$\otimes$分别为:
\begin{equation}
    \begin{aligned}
        a(x) \oplus b(x) = a(x) + b(x)\\
        a(x) \otimes b(x) = (a(x)b(x))_{p(x)}
    \end{aligned}
\end{equation}

请求出$(x+2)\oplus(3x+4)$和$(x+2)\otimes(3x+4)$的值。

\begin{Solution}
    \begin{equation}
        \begin{aligned}
            (x+2)\oplus(3x+4)   &= (x+2) + (3x+4)
                                = 4x + 6\\
                                &= 4x + 1
        \end{aligned}
    \end{equation}
    \begin{equation}
        \begin{aligned}
            (x+2)\otimes(3x+4)  &= ((x+2)(3x+4)) \bmod p(x) = (3x^2 + 4x + 6x + 8) \bmod p(x)\\
                                &= 10x + 2 \\
                                &= 2
        \end{aligned}
    \end{equation}
\end{Solution}


\section{第8题}
设$p=2$,考察$\mathbb{Z}_2[x]$中多项式
\begin{equation}
    p(x) = x^3 + x + 1
\end{equation}

(1) 试说明$p(x)$是$\mathbb{Z}_2[x]$上的不可约多项式;

\begin{Solution}
    $\mathbb{Z}_2[x]$中的一次因式有:$x,x-1$,则有:
    \begin{equation}
        p(0) = 1, p(1) = 1
    \end{equation}
    
    由余元定理的推论可知:$p(x)$无$\mathbb{Z}_2[x]$上的一次因式。

    因此$p(x)$是$\mathbb{Z}_2[x]$上的不可约多项式。
\end{Solution}

(2) $\mathbb{Z}_2[x]_{p(x)}$是含有$2^3=25$个元素的集合,试写出$\mathbb{Z}_2[x]_{p(x)}$的全部元素;

\begin{Solution}
    $\partial^o(p(x))=3$
    ,则剩余类中多项式$r(x)$满足$0\le \partial^o(r(x)) < 3$,因此:
    \begin{equation}
        \begin{aligned}
            &\mathbb{Z}_2[x]_{f(x)} = \left\{ax^2+bx+c|a,b,c \in \mathbb{Z}_2\right\}\\
            =&\{0,1,x,x+1,x^2,x^2+1,x^2+x,x^2+x+1\}
        \end{aligned}
    \end{equation}
\end{Solution}

(3) 在$\mathbb{Z}_2[x]_{p(x)}$上定义加法运算$\oplus$和乘法运算$\otimes$分别为:
\begin{equation}
    \begin{aligned}
        a(x) \oplus b(x) = a(x) + b(x)\\
        a(x) \otimes b(x) = (a(x)b(x))_{p(x)}
    \end{aligned}
\end{equation}

请列出$\mathbb{Z}_2[x]_{p(x)}$关于$\oplus$和$\otimes$这两种运算的运算表。

\begin{Solution}
    由于$\mathbb{Z}_2[x]_{p(x)}\cong\mathbb{Z}_{2^3} = \mathbb{Z}_{8}$,可直接利用$\mathbb{Z}_{8}$的运算表(第5题已得到)
    
    记$f_i(x) = ax^2+bx+c, i = 4a + 2b + c$,给出运算表如下表所示
    
        \centering
        \makeatletter\def\@captype{table}\makeatother\caption{$\mathbb{Z}_2[x]_{p(x)}$上$\oplus$运算表}
        \begin{tabular}{|c|c|c|c|c|c|c|c|c|}
        \hline
        $\oplus$  & $f_0(x)$ &  $f_1(x)$  & $f_2(x)$  & $f_3(x)$  & $f_4(x)$  & $f_5(x)$  & $f_6(x)$  & $f_7(x)$\\ \hline
        $f_0(x)$  & $f_0(x)$ &  $f_1(x)$  & $f_2(x)$  & $f_3(x)$  & $f_4(x)$  & $f_5(x)$  & $f_6(x)$  & $f_7(x)$\\ \hline
        $f_1(x)$  & $f_1(x)$ &  $f_2(x)$  & $f_3(x)$  & $f_4(x)$  & $f_5(x)$  & $f_6(x)$  & $f_7(x)$  & $f_0(x)$\\ \hline
        $f_2(x)$  & $f_2(x)$ &  $f_3(x)$  & $f_4(x)$  & $f_5(x)$  & $f_6(x)$  & $f_7(x)$  & $f_0(x)$  & $f_1(x)$\\ \hline
        $f_3(x)$  & $f_3(x)$ &  $f_4(x)$  & $f_5(x)$  & $f_6(x)$  & $f_7(x)$  & $f_0(x)$  & $f_1(x)$  & $f_2(x)$\\ \hline
        $f_4(x)$  & $f_4(x)$ &  $f_5(x)$  & $f_6(x)$  & $f_7(x)$  & $f_0(x)$  & $f_1(x)$  & $f_2(x)$  & $f_3(x)$\\ \hline
        $f_5(x)$  & $f_5(x)$ &  $f_6(x)$  & $f_7(x)$  & $f_0(x)$  & $f_1(x)$  & $f_2(x)$  & $f_3(x)$  & $f_4(x)$\\ \hline
        $f_6(x)$  & $f_6(x)$ &  $f_7(x)$  & $f_0(x)$  & $f_1(x)$  & $f_2(x)$  & $f_3(x)$  & $f_4(x)$  & $f_5(x)$\\ \hline
        $f_7(x)$  & $f_7(x)$ &  $f_0(x)$  & $f_1(x)$  & $f_2(x)$  & $f_3(x)$  & $f_4(x)$  & $f_5(x)$  & $f_6(x)$\\ \hline
        \end{tabular}

        \centering
        \makeatletter\def\@captype{table}\makeatother\caption{$\mathbb{Z}_2[x]_{p(x)}$上$\otimes$运算表}
        \begin{tabular}{|c|c|c|c|c|c|c|c|c|}
        \hline
        $\otimes$  & $f_0(x)$  & $f_1(x)$  & $f_2(x)$ &  $f_3(x)$ &  $f_4(x)$ &  $f_5(x)$ &  $f_6(x)$ &  $f_7(x)$\\ \hline
        $f_0(x)$  & $f_0(x)$  & $f_0(x)$  & $f_0(x)$ &  $f_0(x)$ &  $f_0(x)$ &  $f_0(x)$ &  $f_0(x)$ &  $f_0(x)$\\ \hline
        $f_1(x)$  & $f_0(x)$  & $f_1(x)$  & $f_2(x)$ &  $f_3(x)$ &  $f_4(x)$ &  $f_5(x)$ &  $f_6(x)$ &  $f_7(x)$\\ \hline
        $f_2(x)$  & $f_0(x)$  & $f_2(x)$  & $f_4(x)$ &  $f_6(x)$ &  $f_0(x)$ &  $f_2(x)$ &  $f_4(x)$ &  $f_6(x)$\\ \hline
        $f_3(x)$  & $f_0(x)$  & $f_3(x)$  & $f_6(x)$ &  $f_1(x)$ &  $f_4(x)$ &  $f_7(x)$ &  $f_2(x)$ &  $f_5(x)$\\ \hline
        $f_4(x)$  & $f_0(x)$  & $f_4(x)$  & $f_0(x)$ &  $f_4(x)$ &  $f_0(x)$ &  $f_4(x)$ &  $f_0(x)$ &  $f_4(x)$\\ \hline
        $f_5(x)$  & $f_0(x)$  & $f_5(x)$  & $f_2(x)$ &  $f_7(x)$ &  $f_4(x)$ &  $f_1(x)$ &  $f_6(x)$ &  $f_3(x)$\\ \hline
        $f_6(x)$  & $f_0(x)$  & $f_6(x)$  & $f_4(x)$ &  $f_2(x)$ &  $f_0(x)$ &  $f_6(x)$ &  $f_4(x)$ &  $f_2(x)$\\ \hline
        $f_7(x)$  & $f_0(x)$  & $f_7(x)$  & $f_6(x)$ &  $f_5(x)$ &  $f_4(x)$ &  $f_3(x)$ &  $f_2(x)$ &  $f_1(x)$\\ \hline
        \end{tabular}
\\
\end{Solution}

\newpage
\chapter{研讨项目1(代数部分)——认识有限域$\mathrm{GF}(p^n)$}

\textbf{目的}:掌握有限域$\mathrm{GF}(p^n)$的结构,以及有限域$\mathrm{GF}(p^n)$上的运算。

\textbf{要求}:讨论有限域$\mathrm{GF}(p^n)$的构造方法,有限域$\mathrm{GF}(p^n)$加法结构和乘法结构,
讨论有限域$\mathrm{GF}(p^n)$上的运算。


在$\mathrm{GF}(2)=\left\{0,1\right\}$的系数域上,以$p(x)=x^4+x^3+1$为模构成有限域$\mathrm{GF}(2^4)$。

(1) 有限域$\mathrm{GF}(2^4)$的特征;
(2) 设$\alpha$为$p(x)$的根,并写出有限域$\mathrm{GF}(2^4)$中元素的四种表示;
(3) 找出所有的共轭根组,并构成相应的最小多项式;
(4) 将所有的最小多项式化简;
(5) 将$x^{16}-x$因式分解;
(6) 求出所有的本原元和本原多项式。



\begin{Solution}
(1) 根据域特征的定义可知,有限域$\mathrm{GF}(2^4)$的特征为$2$。

(2) 由于$\mathrm{GF}(2^4)\cong \mathbb{Z}_2[x]/p(x)$,可记$\mathrm{GF}(2^4)\cong \left\{ax^3+bx^2+cx+d|a,b,c,d\in\mathbb{Z}_2\right\}$

由于$x^4 (\bmod p(x)) = x^3 + 1$,可设$\alpha^4 = \alpha^3 + 1$,得到:\\

{
\centering
\begin{tabular}{ccl}
    \toprule
    幂次表示 & 多项式表示 & 推导过程\\
    \midrule
    $0$ & $0$ &\\
    $\alpha^0 = 1$ &$1$                                 &\\
    $\alpha^1$&$\alpha$                                 &\\   
    $\alpha^2$&$\alpha^2$                               &\\   
    $\alpha^3$&$\alpha^3$                               &\\   
    $\alpha^4$&$\alpha^3 + 1$                           &$\alpha^4 = \alpha^3 + 1$\\   
    $\alpha^5$&$\alpha^3 + \alpha + 1$                  &$\alpha^{5} = \alpha^4\cdot\alpha = (\alpha^3 + 1)\alpha=\alpha^4 + \alpha$\\   
    $\alpha^6$&$\alpha^3 + \alpha^2 + \alpha + 1$       & $\alpha^{6} = \alpha^5\cdot\alpha = (\alpha^3 + \alpha + 1)\alpha = \alpha^4 + \alpha^2 + \alpha$\\   
    $\alpha^7$&$\alpha^2 + \alpha + 1$                  &$\alpha^{7} = \alpha^{6}\cdot\alpha = (\alpha^3 + \alpha^2 + \alpha + 1)\alpha = \alpha^4 + \alpha^3 + \alpha^2 + \alpha$\\   
    $\alpha^8$&$\alpha^3 + \alpha^2 + \alpha$           &$\alpha^{8} = \alpha^{7}\cdot\alpha = (\alpha^2 + \alpha + 1)\alpha$\\   
    $\alpha^9$&$\alpha^2 + 1$                           &$\alpha^{9} = \alpha^{8}\cdot\alpha = (\alpha^3 + \alpha^2 + \alpha) \alpha = \alpha^4 + \alpha^3 +\alpha^2$\\   
    $\alpha^{10}$&$\alpha^3 + \alpha$                   &$\alpha^{10} = \alpha^9\cdot\alpha = (\alpha^2 + 1)\alpha$ \\       
    $\alpha^{11}$&$\alpha^3 + \alpha^2 + 1$             &$\alpha^{11} = \alpha^{10}\cdot\alpha = (\alpha^3 + \alpha) \alpha = \alpha^4 + \alpha^2$\\       
    $\alpha^{12}$&$\alpha + 1$                          &$\alpha^{12}=\alpha^{10}\cdot\alpha = (\alpha^3 + \alpha^2 + 1)\alpha=\alpha^4+\alpha^3+\alpha$\\       
    $\alpha^{13}$&$\alpha^2 + \alpha$                   &$\alpha^{13}=\alpha^{11}\cdot\alpha = (\alpha + 1)\alpha$\\       
    $\alpha^{14}$&$\alpha^3 + \alpha^2$                 &$\alpha^{14}=\alpha^{13}\cdot\alpha = (\alpha^2 + \alpha)\alpha$\\       
    \bottomrule
\end{tabular}}


\newpage
可发现通过生成元循环生成的循环域$\left<\alpha\right> + \{0\}$在模$p(x)$后得到的有限域$(\left<\alpha\right>+ \{0\})/p(x)$与剩余类$\mathbb{Z}_2[x]/p(x)$存在一一映射,
即$(\left<\alpha\right>+ \{0\})/p(x)\cong \mathbb{Z}_2[x]/p(x)$。因此该方法可得到有限域$\mathrm{GF}(2^4)$的另一个同构域。
此外我们将剩余类的系数$a,b,c,d$提取为向量形式,就可得到有限域$\mathrm{GF}(2^4)$的四种表示法:
\begin{center}
    \begin{tabular}{cccc}
        \toprule
        剩余类表示 & 多项式表示 & 幂次表示 & 矢量表示\\
        \midrule
        $0$                             &$0$                                    &$0$                 & $0000$  \\
        $1$                             &$1$                                    &$\alpha^0 = 1$      & $0001$  \\
        $x$                             &$\alpha$                               &$\alpha^1$          & $0010$  \\
        $x^2$                           &$\alpha^2$                             &$\alpha^2$          & $0100$  \\
        $x^3$                           &$\alpha^3$                             &$\alpha^3$          & $1000$  \\
        $x^3 + 1$                       &$\alpha^3 + 1$                         &$\alpha^4$          & $1001$  \\
        $x^3 + x + 1$                   &$\alpha^3 + \alpha + 1$                &$\alpha^5$          & $1011$  \\
        $x^3 + x^2 + x + 1$             &$\alpha^3 + \alpha^2 + \alpha + 1$     &$\alpha^6$          & $1111$  \\
        $x^2 + x + 1$                   &$\alpha^2 + \alpha + 1$                &$\alpha^7$          & $0111$  \\
        $x^3 + x^2 + x$                 &$\alpha^3 + \alpha^2 + \alpha$         &$\alpha^8$          & $1110$  \\
        $x^2 + 1$                       &$\alpha^2 + 1$                         &$\alpha^9$          & $0101$  \\
        $x^3 + x$                       &$\alpha^3 + \alpha$                    &$\alpha^{10}$       & $1010$  \\
        $x^3 + x^2 + 1$                 &$\alpha^3 + \alpha^2 + 1$              &$\alpha^{11}$       & $1101$  \\
        $x + 1$                         &$\alpha + 1$                           &$\alpha^{12}$       & $0011$  \\
        $x^2 + x$                       &$\alpha^2 + \alpha$                    &$\alpha^{13}$       & $0110$  \\
        $x^3 + x^2$                     &$\alpha^3 + \alpha^2$                  &$\alpha^{14}$       & $1100$  \\
        \bottomrule
    \end{tabular}
\end{center}

(3) 对于有限域$\mathrm{GF}(2^4)$中的任意元素$\beta$,迭代计算$\beta, \beta^p, \beta^{p^2}, \cdots, \beta^{p^k} = \beta$,
即可得到所有的共轭根组,同时可构成对应的最小多项式。
\begin{center}
    \begin{tabular}{ll}
        \toprule
        共轭根组 & 最小多项式 \\
        \midrule
        $\left\{ 0\right\}$                                                             &$m(x)=x$   \\
        $\left\{ 1\right\}$                                                             &$m_0(x)=x+1$   \\
        $\left\{ \alpha, \alpha^2, \alpha^4, \alpha^8\right\}$                          &$m_1(x)=(x-\alpha)(x-\alpha^2)(x-\alpha^4)(x-\alpha^8)$   \\
        $\left\{ \alpha^{3}, \alpha^{6}, \alpha^{12}, \alpha^{9}\right\}$               &$m_3(x)=(x-\alpha^3)(x-\alpha^6)(x-\alpha^9)(x-\alpha^{12})$   \\
        $\left\{ \alpha^{5}, \alpha^{10}\right\}$                                       &$m_5(x)=(x-\alpha^5)(x-\alpha^{10})$   \\
        $\left\{ \alpha^{7}, \alpha^{14}, \alpha^{13}, \alpha^{11}\right\}$             &$m_7(x)=(x-\alpha^7)(x-\alpha^{11})(x-\alpha^{13})(x-\alpha^{14})$   \\
        \bottomrule
    \end{tabular}
\end{center}

(4) 下面对(3)中的所有最小多项式化简。

    $
    \begin{aligned}
        m(x)    &= x\\
        m_0(x)  &= x+1\\
        m_1(x)  &= (x-\alpha)(x-\alpha^2)(x-\alpha^4)(x-\alpha^8)\\
                &= (x^2 - (\alpha + \alpha^2)x + \alpha^3)(x^2 - (\alpha^4 + \alpha^8)x + \alpha^{12})\\
                &= x^4 - (\alpha + \alpha^2 + \alpha^4 + \alpha^8)x^3 + (\alpha^3 + \alpha^{12} + \alpha^5 + \alpha^9 + \alpha^6 + \alpha^{10})x^2\\
                &~~~~ - (\alpha^{13} + \alpha^{14} + \alpha^{7} + \alpha^{11})x + \alpha^{15}\\
                &= x^4 - (0010+0100+1001+1110)x^3 + (1000 + 1011 + 1111 \\
                &~~~~ + 1010 + 1010 + 0011)x^2 - (0111 + 1101 + 0110 + 1100)x + 1\\
                &= x^4 - 0001x^3 + 0000 x^2 - 0000x + 1\\
                &= x^4 + x^3 + 1\\
        m_3(x)  &= (x-\alpha^3)(x-\alpha^6)(x-\alpha^9)(x-\alpha^{12})\\
                &= (x^2 - (\alpha^3 + \alpha^6)x + \alpha^9)(x^2 - (\alpha^9 + \alpha^{12})x + \alpha^{6})\\
                &= x^4 - (\alpha^3 + \alpha^6 + \alpha^9 + \alpha^{12})x^3 + (\alpha^{9}+\alpha^{6}+\alpha^{12}+\alpha^{0}+\alpha^{0}+\alpha^{18})x^2\\
                &~~~~ - (\alpha^{9} + \alpha^{12} + \alpha^{3} + \alpha^{6})x + \alpha^{15}\\
                &= x^4 - (1000+1111+0101+0011)x^3 + (0101 +1111 \\
                &~~~~ + 0011 + 1000)x^2 - (0101 + 0011 + 1000 + 1111)x + 1\\
                &= x^4 - 0001x^3 + 0001 x^2 - 0001x + 1\\
                &= x^4 + x^3 + x^2 + x + 1\\
        m_5(x)  &= (x-\alpha^5)(x-\alpha^{10})= x^2 - (\alpha^5 + \alpha^{10})x + \alpha^0\\
                &= x^2 - (1011 + 1010)x + 1 = x^2 - 0001x + 1\\
                &= x^2 + x + 1\\
        m_7 (x) &= (x-\alpha^7)(x-\alpha^{11})(x-\alpha^{13})(x-\alpha^{14})\\
                &= (x^2 - (\alpha^7 + \alpha^{11})x + \alpha^3)(x^2 - (\alpha^{13} + \alpha^{14})x + \alpha^{12})\\
                &= x^4 - (\alpha^7 + \alpha^{11} + \alpha^{13} + \alpha^{14})x^3 + (\alpha^{3}+\alpha^{12}+\alpha^{5}+\alpha^{6}+\alpha^{9}+\alpha^{10})x^2\\
                &~~~~ - (\alpha^{4} + \alpha^{8} + \alpha^{1} + \alpha^{2})x + \alpha^{15}\\
                &= x^4 - (0111 + 1101 + 0110 + 1100)x^3 + (1000 + 0011 + 1011 \\
                &~~~~ + 1111 + 0101 + 1010)x^2 - (1001 + 1110 + 0010 + 0100)x + 1\\
                &= x^4 - 0000x^3 + 0000 x^2 - 0001x + 1\\
                &= x^4 + x + 1\\
    \end{aligned}
    $

\newpage
综上所述,有:
\begin{center}
    \begin{tabular}{ll}
        \toprule
        共轭根组 & 最小多项式 \\
        \midrule
        $\left\{ 0\right\}$             &$m(x)=x$   \\
        $\left\{ 1\right\}$             &$m_0(x)=x+1$   \\
        $\left\{ \alpha, \alpha^2, \alpha^4, \alpha^8\right\}$             &$m_1(x)=x^4 + x^3 + 1$   \\
        $\left\{ \alpha^{3}, \alpha^{6}, \alpha^{12}, \alpha^{9}\right\}$             &$m_3(x)=x^4 + x^3 + x^2 + x + 1$   \\
        $\left\{ \alpha^{5}, \alpha^{10}\right\}$             &$m_5(x)=x^2 + x + 1$   \\
        $\left\{ \alpha^{7}, \alpha^{14}, \alpha^{13}, \alpha^{11}\right\}$             &$m_7(x)=x^4 + x + 1$   \\
        \bottomrule
    \end{tabular}
\end{center}

(5) 由Fermat小定理,对于非零元$\alpha \in \mathrm{GF}(16)$有:
\begin{equation}
    x^{16} - x = \prod_{\alpha \in \mathrm{GF}(16)} (x-\alpha)
\end{equation}

因此可对$x^{16}- x$因式分解,并对其因式重排列:\\

$
\begin{aligned}
    x^{16}- x   &= x (x+1) \left[(x-\alpha)(x-\alpha^2)(x-\alpha^4)(x-\alpha^8)\right]\left[(x-\alpha^3)(x-\alpha^6)(x-\alpha^9)(x-\alpha^{12})\right]\\
                &~~~~\left[(x-\alpha^5)(x-\alpha^{10})\right]\left[(x-\alpha^7)(x-\alpha^{11})(x-\alpha^{13})(x-\alpha^{14})\right]\\
                &= m(x)m_0(x)m_1(x)m_3(x)m_5(x)m_7(x)\\
                &= x(x+1)(x^4 + x^3 + 1)(x^4 + x^3 + x^2 + x + 1)(x^2 + x + 1)(x^4 + x + 1)\\
\end{aligned}
$\\

(6) 编写简单脚本计算元素阶
\begin{lstlisting}[language = Python]
for i in range(15):
    order = 1
    iter = (i + i) % 15
    while iter != i:
        iter += i
        iter %= 15
        order += 1
    print(f'{i}\'s order: {order}')
\end{lstlisting}
得到:

\begin{itemize}
    


    \item $\forall a \in                                                            
\left\{ 1\right\}$,$ord(a) = 1$

\item $\forall a \in 
\left\{ \alpha, \alpha^2, \alpha^4, \alpha^8\right\}                                                
\cup \left\{ \alpha^{7}, \alpha^{14}, \alpha^{13}, \alpha^{11}\right\}$,$ord(a) = 15$

\item $\forall a \in\left\{ \alpha^{3}, \alpha^{6}, \alpha^{12}, \alpha^{9}\right\}$       
,$ord(a) = 5$

\item $\forall a \in\left\{ \alpha^{5}, \alpha^{10}\right\}$
,$ord(a) = 3$

\end{itemize}

因此本原元即为$\forall a \in \left\{ \alpha, \alpha^2, \alpha^4, \alpha^8\right\}                                                
\cup \left\{ \alpha^{7}, \alpha^{14}, \alpha^{13}, \alpha^{11}\right\}$

本原多项式为:$m_1(x)=x^4 + x^3 + 1, m_7(x)=x^4 + x + 1$
\end{Solution}

\newpage
\chapter{第六周作业}
\section{第9题}
将域$\mathbb{F}_3$上的$4\times 4$矩阵$\bm{A}$化为与之行等价的阶梯形矩阵$\bm{A}_0$,
\begin{equation}
    \bm{A} = \left(
        \begin{smallmatrix}
            1 &2 &0 &2\\
            2 &1 &1 &0\\
            2 &2 &1 &1\\
            0 &1 &2 &2\\
        \end{smallmatrix}
    \right)
\end{equation}
并求出矩阵$\bm{A}$的秩。

\begin{Solution}
    由题意,对矩阵$\bm{A}$做如下初等行变换:\\

    $
    \begin{aligned}
        \bm{A} &= \left(
        \begin{smallmatrix}
            1 &2 &0 &2\\
            2 &1 &1 &0\\
            2 &2 &1 &1\\
            0 &1 &2 &2\\
        \end{smallmatrix}
    \right)
    \sim
    \left(
        \begin{smallmatrix}
            2 &1 &0 &1\\
            2 &1 &1 &0\\
            2 &2 &1 &1\\
            0 &1 &2 &2\\
        \end{smallmatrix}
    \right)
    \sim
    \left(
        \begin{smallmatrix}
            2 &1 &0 &1\\
            0 &0 &1 &2\\
            0 &1 &1 &0\\
            0 &1 &2 &2\\
        \end{smallmatrix}
    \right)
    \sim
    \left(
        \begin{smallmatrix}
            2 &1 &0 &1\\
            0 &1 &1 &0\\
            0 &1 &2 &2\\
            0 &0 &1 &2\\
        \end{smallmatrix}
    \right)\\
    &\sim
    \left(
        \begin{smallmatrix}
            2 &1 &0 &1\\
            0 &1 &1 &0\\
            0 &0 &1 &2\\
            0 &0 &1 &2\\
        \end{smallmatrix}
    \right)
    \sim
    \left(
        \begin{smallmatrix}
            2 &1 &0 &1\\
            0 &1 &1 &0\\
            0 &0 &1 &2\\
            0 &0 &0 &0\\
        \end{smallmatrix}
    \right)= \bm{A}_0
    \end{aligned} 
    $\\

    因此,$rank(\bm{A})=rank(\bm{A}_0) = 3$。
\end{Solution}

\section{第10题}
将域$\mathbb{F}_2$上的$8\times 8$矩阵$\bm{A}$化为与之行等价的阶梯形矩阵$\bm{A}_0$,
\begin{equation}
    \bm{A} = \left(
        \begin{smallmatrix}
            1 &0 &1 &1 &0 &1 &1 &0\\
            1 &0 &0 &0 &0 &0 &0 &0\\
            0 &1 &0 &1 &0 &1 &0 &1\\
            1 &1 &1 &1 &1 &1 &1 &0\\
            1 &0 &0 &0 &0 &1 &1 &1\\
            0 &0 &0 &0 &0 &0 &0 &1\\
            0 &0 &0 &0 &0 &1 &1 &1\\
            0 &1 &1 &0 &1 &0 &1 &0\\
        \end{smallmatrix}
    \right)
\end{equation}
并求出矩阵$\bm{A}$的秩。

\begin{Solution}
    由题意,对矩阵$\bm{A}$做如下初等行变换:\\

    $
    \begin{aligned}
        \bm{A} &= \left(
        \begin{smallmatrix}
            1 &0 &1 &1 &0 &1 &1 &0\\
            1 &0 &0 &0 &0 &0 &0 &0\\
            0 &1 &0 &1 &0 &1 &0 &1\\
            1 &1 &1 &1 &1 &1 &1 &0\\
            1 &0 &0 &0 &0 &1 &1 &1\\
            0 &0 &0 &0 &0 &0 &0 &1\\
            0 &0 &0 &0 &0 &1 &1 &1\\
            0 &1 &1 &0 &1 &0 &1 &0\\
        \end{smallmatrix}
    \right)
    \sim
    \left(
        \begin{smallmatrix}
            1 &0 &1 &1 &0 &1 &1 &0\\
            0 &0 &1 &1 &0 &1 &1 &0\\
            0 &1 &0 &1 &0 &1 &0 &1\\
            0 &1 &0 &0 &1 &0 &0 &0\\
            0 &0 &1 &1 &0 &0 &0 &1\\
            0 &0 &0 &0 &0 &0 &0 &1\\
            0 &0 &0 &0 &0 &1 &1 &1\\
            0 &1 &1 &0 &1 &0 &1 &0\\
        \end{smallmatrix}
    \right)
    \sim
    \left(
        \begin{smallmatrix}
            1 &0 &1 &1 &0 &1 &1 &0\\
            0 &1 &1 &0 &1 &0 &1 &0\\
            0 &1 &0 &1 &0 &1 &0 &1\\
            0 &1 &0 &0 &1 &0 &0 &0\\
            0 &0 &1 &1 &0 &0 &0 &1\\
            0 &0 &0 &0 &0 &0 &0 &1\\
            0 &0 &0 &0 &0 &1 &1 &1\\
            0 &0 &1 &1 &0 &1 &1 &0\\
        \end{smallmatrix}
    \right)\\
    &\sim
    \left(
        \begin{smallmatrix}
            1 &0 &1 &1 &0 &1 &1 &0\\
            0 &1 &1 &0 &1 &0 &1 &0\\
            0 &0 &1 &1 &1 &1 &1 &1\\
            0 &0 &1 &1 &0 &1 &1 &0\\
            0 &0 &1 &0 &0 &0 &1 &0\\
            0 &0 &1 &1 &0 &0 &0 &1\\
            0 &0 &0 &0 &0 &0 &0 &1\\
            0 &0 &0 &0 &0 &1 &1 &1\\
        \end{smallmatrix}
    \right)
    \sim
    \left(
        \begin{smallmatrix}
            1 &0 &1 &1 &0 &1 &1 &0\\
            0 &1 &1 &0 &1 &0 &1 &0\\
            0 &0 &1 &1 &1 &1 &1 &1\\
            0 &0 &0 &0 &1 &0 &0 &1\\
            0 &0 &0 &1 &1 &1 &0 &1\\
            0 &0 &0 &0 &1 &1 &1 &0\\
            0 &0 &0 &0 &0 &1 &1 &1\\
            0 &0 &0 &0 &0 &0 &0 &1\\
        \end{smallmatrix}
    \right)
    \sim
    \left(
        \begin{smallmatrix}
            1 &0 &1 &1 &0 &1 &1 &0\\
            0 &1 &1 &0 &1 &0 &1 &0\\
            0 &0 &1 &1 &1 &1 &1 &1\\
            0 &0 &0 &1 &1 &1 &0 &1\\
            0 &0 &0 &0 &1 &1 &1 &0\\
            0 &0 &0 &0 &0 &1 &1 &1\\
            0 &0 &0 &0 &0 &0 &0 &1\\
            0 &0 &0 &0 &0 &0 &0 &0\\
        \end{smallmatrix}
    \right)=\bm{A_0}
    \end{aligned}
    $\\

    因此,$rank(\bm{A})=rank(\bm{A}_0) = 7$。
\end{Solution}

\newpage
\chapter{第七周作业}

\section{第11题}

求出齐次线性方程组$\bm{A}\bm{x}'=\bm{0}'$的解空间的一组基
\begin{equation}
    \bm{A} = \left(
        \begin{matrix}
            1 &0 &1 &0 &0 &1\\
            1 &1 &0 &1 &1 &0\\
            0 &1 &0 &1 &1 &0\\
        \end{matrix}
    \right)
\end{equation}

\begin{Solution}
    认为$\bm{A} = \mathbb{F}_2^{3\times 6}$,由题意,对矩阵$\bm{A}$做如下初等行变换:\\

    $
    \begin{aligned}
        \bm{A} &= \left(
        \begin{matrix}
            1 &0 &1 &0 &0 &1\\
            1 &1 &0 &1 &1 &0\\
            0 &1 &0 &1 &1 &0\\
        \end{matrix}
    \right)
    \sim
    \left(
        \begin{matrix}
            1 &0 &1 &0 &0 &1\\
            0 &1 &1 &1 &1 &1\\
            0 &1 &0 &1 &1 &0\\
        \end{matrix}
    \right)
    \sim
    \left(
        \begin{matrix}
            1 &0 &0 &0 &0 &0\\
            0 &1 &0 &1 &1 &0\\
            0 &0 &1 &0 &0 &1\\
        \end{matrix}
    \right)
    = \bm{A}_0
    \end{aligned}
    $\\

    由于$\bm{A}\sim \bm{A}_0$,因此线性方程组$\bm{A}_0\bm{x}'=\bm{0}$与$\bm{A}\bm{x}'=\bm{0}$同解。

    取前3列为主元,则有:
    \begin{equation}
        \left\{
        \begin{aligned}
            &x_1 = 0\\
            &x_2 = k_1 + k_2\\
            &x_3 = k_3\\
        \end{aligned}
        \right.
    \end{equation}

    得到基础解系$\xi_1=(0 10100)',\xi_2=(010010)',\xi_3=(001001)'$。

    因此$\bm{x} = k_1\xi_1 + k_2\xi_2 + k_3\xi_3,~k_1,k_2,k_3\in \mathbb{F}_2$,
    
    且$\xi_1, \xi_2, \xi_3$恰为解空间的一组基。
\end{Solution}


\newpage
\section{第12题}

设码$C=\left\{(1 0 1 0 0), (0 1 0 1 0), (0 1 1 0 1), (1 0 0 1 1)\right\}$,

(1) 计算$C$中所有码字的两两Hamming距离;

(2) 指出$C$是可以检几错的检错码,并且指出$C$是可以纠几错的纠错码。

\begin{Solution}
    (1) 设 $d_{i,j}$表示$C_i$和$C_j$的Hamming距离,则:

    $d_{1,2} = 4,~d_{1,3} = 3,~d_{1,4} = 3,~d_{2,3} = 3,~d_{2,4} = 3,~d_{3,4} = 4$

    (2) 此时$C$的极小距离$d_{\min}$即为$3$,根据纠错码基本定理:

    可检错误:$d_{\min} - 1 = 3 - 1  = 2$

    可纠错误:$\lfloor (d_{\min} -1)/ 2\rfloor = \lfloor (3 -1)/ 2\rfloor = 1$

    即$C$是可以检$2$错,纠$1$错的纠错码。

\end{Solution}

\section{第13题}

设$C$是一个二元$(6,3)$线性码,其生成矩阵为:
\begin{equation}
    \bm{G} = \left(
        \begin{matrix}
            1 &0 &1 &1 &0 &0\\
            0 &1 &0 &0 &1 &0\\
            0 &0 &1 &1 &1 &1\\
        \end{matrix}
    \right)
\end{equation}

(1) 求出$C$的全部码字;

(2) 求出$C$的校验矩阵$\bm{H}$。

\begin{Solution}
    (1)$\sigma(a_0,a_1,a_2) = a_0v_0 + a_1v_1 + a_2v_2 = (a_0,a_1,a_2)\bm{G}$
    
    因此$C$中共有$8$个码字:

    $
    \begin{aligned}
        &\sigma(000) = 000000,~
        \sigma(001) = 001111,~
        \sigma(010) = 010010,~
        \sigma(011) = 011101,\\
        &\sigma(100) = 101100,~
        \sigma(101) = 100011,~
        \sigma(110) = 111110,~
        \sigma(111) = 110001\\
    \end{aligned}
    $\\

    (2) 通过初等行变换可得到与C等价的系统码的生成矩阵$\bm{G}_0$:\\

    $
    \begin{aligned}
        \bm{G} &= \left(
        \begin{matrix}
            1 &0 &1 &1 &0 &0\\
            0 &1 &0 &0 &1 &0\\
            0 &0 &1 &1 &1 &1\\
        \end{matrix}
    \right)
    \sim
    \left(
        \begin{matrix}
            1 &0 &0 &0 &1 &1\\
            0 &1 &0 &0 &1 &0\\
            0 &0 &1 &1 &1 &1\\
        \end{matrix}
    \right)
    = \bm{G}_0
    \end{aligned}
    $\\

    由于等价码的生成矩阵行等价,因此其对应解空间相同。因此码$C$校验矩阵$\bm{H}$即为系统码$C_0$的校验矩阵:
    \begin{equation}
        \bm{H} = \bm{H}_0=
        \left(
            \begin{matrix}
                0 &0 &1 &1 &0 &0\\
                1 &1 &1 &0 &1 &0\\
                1 &0 &1 &0 &0 &1\\
            \end{matrix}
        \right)
    \end{equation}

\end{Solution}